\documentclass[DM,lsstdraft,STS,toc]{lsstdoc}
\usepackage{enumitem}
\usepackage{booktabs}
\input meta.tex

\begin{document}

\providecommand{\tightlist}{%
  \setlength{\itemsep}{0pt}\setlength{\parskip}{0pt}}

\def\product{LSST Data Management}

\setDocCompact{true}

\title[Test Spec for \product]{\product~Science Acceptance Test Specification}

\author{L.P. Guy, M. Wood-Vasey, E. Bellm, J.F. Bosch,
G.P. Dubois-Felsmann,  M.L. Graham, R. Gruendl, 
K.S. Krughoff, K.-T. Lim, R.H. Lupton, C. Slater}
\setDocRef{\lsstDocType-\lsstDocNum}
\setDocDate{\vcsdate}

\setDocAbstract {
This document describes the detailed acceptance test specification for the \product{} System.
}

% Most recent last
\setDocChangeRecord{%
	\addtohist{1.0}{2018-06-12}{Initial of draft version.}{gcm}
	\addtohist{1.1}{2018-06-18}{Updated author list, added sections 1 and 2 }{L. Guy}
}

\setDocCurator{Leanne Guy}
\setDocUpstreamLocation{\url{https://github.com/lsst/ldm-639}}
\setDocUpstreamVersion{\vcsrevision}

\maketitle

\section{Introduction}
\label{sec:intro}

This document specifies the science acceptance test procedure for the \product{} System.

\subsection{Objectives}
\label{sec:objectives}

This document describes the test cases required to validate the Data Management System requirements
described in the LSST DM Subsystem Requirements document \citeds{LSE-61}.  
A full description of this product is provided in the Data Management System Design document, \citeds{LDM-148} with 
the science requirements  detailed in  the LSST Science Requirements Document \citeds{LPM-17}.
% and \citeds{LSE-30}

It identifies test cases and procedures for the tests as well as the pass/fail criteria for each test.
\subsection{Scope}
\label{sec:scope}

This document provides the science acceptance test plan for the whole Data Management System (DMS), as described 
by the Data Management System Requirements in \citeds{LSE-61}.

\subsection{Applicable Documents}
\label{sec:docs}

\addtocounter{table}{-1}
	
\begin{tabular}[htb]{l l}
\citeds{LPM-17} & LSST Science Requirements Document \\ 
\citeds{LDM-148} & LSST Data Management System Design \\ 
\citeds{LDM-294} & LSST DM Organization \& Management \\
% This document will be merged to LDM-61
% \citeds{LDM-502} & The Measurement and Verification of DM Key Performance Metrics \\
\citeds{LDM-503} & LSST DM Test Plan \\
\citeds{LSE-61}  & LSST DM Subsystem Requirements \\
\citeds{LSE-163} & LSST Data Products Definition Document \\
\citeds{LDM-151} & LSST DM Science Pipelines Design \\
\citeds{LSE-180} & Level 2 Photometric Calibration for the LSST Survey \\
\citeds{LSE-30} & LSST Observatory System Specifications \\ 
\end{tabular}

\subsection{References\label{sect:references}}
\renewcommand{\refname}{}
\bibliography{lsst,refs,books,refs_ads}

%\subsection{Definitions, acronyms, and abbreviations \label{sect:acronyms}} % include acronyms.tex generated by the acronyms.csh (GaiaTools)
%\addtocounter{table}{-1}
\begin{longtable}{|l|p{0.8\textwidth}|}\hline
\textbf{Acronym} & \textbf{Description}  \\\hline

AP & Alerts Production \\\hline
C & Specific programming language (also called ANSI-C) \\\hline
CPP & C++ Programming language \\\hline
DAC & Data Access Center \\\hline
DB & DataBase \\\hline
DBB & Data BackBone \\\hline
DM & Data Management \\\hline
DMCCB & DM Change Control Board \\\hline
DMS & Data Management Sub-system \\\hline
DR & Data Release \\\hline
DRP & Data Release Production \\\hline
EFD & Engineering Facilities Database \\\hline
IT & Integration Test \\\hline
IVOA & International Virtual-Observatory Alliance \\\hline
K & Kelvin; SI unit of temperature \\\hline
LAN & Local Area Network \\\hline
LDM & LSST Data Management (handle for controlled documents) \\\hline
LPM & LSST Project Management (Document Handle) \\\hline
LSE & LSST Systems Engineering (Document Handle) \\\hline
LSP & LSST Science Platform \\\hline
LSST & Large Synoptic Survey Telescope \\\hline
M & Mega; SI units prefix for 1E6 \\\hline
MOPS & Moving Object Pipeline System \\\hline
OCS & Observatory Control System \\\hline
PDAC & Prototype Data Access Center \\\hline
S & Strip (CCD chip along-scan coordinate identifier in focal plane) \\\hline
SODA & SCOS ORATOS Distributed Access \\\hline
SQL & Structured Query Language \\\hline
STS & System Test Specification \\\hline
W & Watt; SI unit of power \\\hline
p & pico; SI units prefix for 1E-12 \\\hline
\end{longtable}



%----------------------------------------------------
% TASK IDENTIFICATION - APPROACH
%----------------------------------------------------
\section{Approach}
\label{sec:approach}

The approach is based on the Jira Test Management Plugin. 

\subsection{Tasks and criteria}
\label{sec:tasks}

This document describes the science acceptance tests for the whole integrated Data Management System, 
with focus on the scientific quality of the data products produced by the LSST data management system and 
whether they satisfy the requirements described in \citeds{LSE-61}

\subsection{Features to be tested}
\label{sec:feat2test}

All top-level requirements for the LSST Data Management System described in \citeds{LSE-61} are to be tested, including 
\begin{itemize}
\item Data Products
\item Alert, Calibration and Data Release Production
\item LSST science pipeline software and middleware
\item LSST facilities including the data archive, base, summit, and the communications between them to accept science and engineering data
\end{itemize}

\subsection{Features not to be tested}
\label{sec:featnot2test}

This document does not describe facilities for periodically generating or collecting key performance metrics (KPMs), 
except insofar as those KPMs are incidentally measured as part of executing the documented test cases. 

\subsection{Pass/fail criteria}
\label{sec:passfail}

The results of all tests will be assessed using the criteria described in
\citeds{LDM-503} \S4.

Note that, when executing pipelines, tasks or individual algorithms, any unexplained or un- expected errors or warnings 
appearing in the associated log or on screen output must be described in the documentation for the system under test. 
Any warning or error for which this is not the case must be filed as a software problem report and filed with the DMCCB.


\subsection{Suspension criteria and resumption requirements}
\label{suspension}

Refer to individual test cases where applicable.

\subsection{Naming convention}

The naming convention applied is derived from Jira Test Management Plugin, specifically: 
\begin{description}

  \item[LVV-XXX]{: Are the Verification Elements, where XXX is the Verification Element identifier. A verification Element can have many Test Cases. }
  \item[LVV-TYY]{: Are the Test Cases. Each Test Case is associated with a Verification Element, where YY is the Test Case identifier)}

\end{description}

The Verification Elements are drawn from \citeds{LSE-61} requirements.  
\newpage
\section{Test Cases Summary}\label{test-cases-summary}

\begin{longtable}[]{p{3cm}p{13cm}}
\toprule
Jira Id & Test Name\tabularnewline
\midrule
\endhead
\protect\hyperlink{lvv-t120---verify-implementation-of-software-framework-for-level-3-catalog-processing-dms-req-0125}{LVV-T120}
&
\href{https://jira.lsstcorp.org/secure/Tests.jspa\#/testCase/LVV-T120}{Verify
implementation of Software framework for Level 3 catalog processing
(DMS-REQ-0125)}\tabularnewline
\protect\hyperlink{lvv-t121---verify-implementation-of-software-framework-for-level-3-image-processing-dms-req-0128}{LVV-T121}
&
\href{https://jira.lsstcorp.org/secure/Tests.jspa\#/testCase/LVV-T121}{Verify
implementation of Software framework for Level 3 image processing
(DMS-REQ-0128)}\tabularnewline
\protect\hyperlink{lvv-t123---verify-implementation-of-access-controls-of-level-3-data-products-dms-req-0340}{LVV-T123}
&
\href{https://jira.lsstcorp.org/secure/Tests.jspa\#/testCase/LVV-T123}{Verify
implementation of Access Controls of Level 3 Data Products
(DMS-REQ-0340)}\tabularnewline
\protect\hyperlink{lvv-t125---verify-implementation-of-simulated-data-dms-req-0009}{LVV-T125}
&
\href{https://jira.lsstcorp.org/secure/Tests.jspa\#/testCase/LVV-T125}{Verify
implementation of Simulated Data (DMS-REQ-0009)}\tabularnewline
\protect\hyperlink{lvv-t126---verify-implementation--image-differencing-dms-req-0032}{LVV-T126}
&
\href{https://jira.lsstcorp.org/secure/Tests.jspa\#/testCase/LVV-T126}{Verify
implementation Image Differencing (DMS-REQ-0032)}\tabularnewline
\protect\hyperlink{lvv-t127---verify-implementation-of-provide-source-detection-software-dms-req-0033}{LVV-T127}
&
\href{https://jira.lsstcorp.org/secure/Tests.jspa\#/testCase/LVV-T127}{Verify
implementation of Provide Source Detection Software
(DMS-REQ-0033)}\tabularnewline
\protect\hyperlink{lvv-t128---verify-implementation-provide-astrometric-model-dms-req-0042}{LVV-T128}
&
\href{https://jira.lsstcorp.org/secure/Tests.jspa\#/testCase/LVV-T128}{Verify
implementation Provide Astrometric Model (DMS-REQ-0042)}\tabularnewline
\protect\hyperlink{lvv-t129---verify-implementation-of-provide-calibrated-photometry-dms-req-0043}{LVV-T129}
&
\href{https://jira.lsstcorp.org/secure/Tests.jspa\#/testCase/LVV-T129}{Verify
implementation of Provide Calibrated Photometry
(DMS-REQ-0043)}\tabularnewline
\protect\hyperlink{lvv-t130---verify-implementation-of-enable-a-range-of-shape-measurement-approaches-dms-req-0052}{LVV-T130}
&
\href{https://jira.lsstcorp.org/secure/Tests.jspa\#/testCase/LVV-T130}{Verify
implementation of Enable a Range of Shape Measurement Approaches
(DMS-REQ-0052)}\tabularnewline
\protect\hyperlink{lvv-t132---verify-implementation-of-pre-cursor-and-real-data-dms-req-0296}{LVV-T132}
&
\href{https://jira.lsstcorp.org/secure/Tests.jspa\#/testCase/LVV-T132}{Verify
implementation of Pre-cursor, and Real Data
(DMS-REQ-0296)}\tabularnewline
\protect\hyperlink{lvv-t133---verify-implementation-of-provide-beam-projector-coordinate-calculation-software-dms-req-0351}{LVV-T133}
&
\href{https://jira.lsstcorp.org/secure/Tests.jspa\#/testCase/LVV-T133}{Verify
implementation of Provide Beam Projector Coordinate Calculation Software
(DMS-REQ-0351)}\tabularnewline
\protect\hyperlink{lvv-t134---verify-implementation-of-provide-image-access-services-dms-req-0065}{LVV-T134}
&
\href{https://jira.lsstcorp.org/secure/Tests.jspa\#/testCase/LVV-T134}{Verify
implementation of Provide Image Access Services
(DMS-REQ-0065)}\tabularnewline
\protect\hyperlink{lvv-t135---verify-implementation-of-provide-data-access-services-dms-req-0155}{LVV-T135}
&
\href{https://jira.lsstcorp.org/secure/Tests.jspa\#/testCase/LVV-T135}{Verify
implementation of Provide Data Access Services
(DMS-REQ-0155)}\tabularnewline
\protect\hyperlink{lvv-t136---verify-implementation-of-data-product-and-raw-data-access-dms-req-0298}{LVV-T136}
&
\href{https://jira.lsstcorp.org/secure/Tests.jspa\#/testCase/LVV-T136}{Verify
implementation of Data Product and Raw Data Access
(DMS-REQ-0298)}\tabularnewline
\protect\hyperlink{lvv-t137---verify-implementation-of-data-product-ingest-dms-req-0299}{LVV-T137}
&
\href{https://jira.lsstcorp.org/secure/Tests.jspa\#/testCase/LVV-T137}{Verify
implementation of Data Product Ingest (DMS-REQ-0299)}\tabularnewline
\protect\hyperlink{lvv-t143---verify-implementation-of-provide-pipeline-construction-services-dms-req-0158}{LVV-T143}
&
\href{https://jira.lsstcorp.org/secure/Tests.jspa\#/testCase/LVV-T143}{Verify
implementation of Provide Pipeline Construction Services
(DMS-REQ-0158)}\tabularnewline
\protect\hyperlink{lvv-t144---verify-implementation-of-task-specification-dms-req-0305}{LVV-T144}
&
\href{https://jira.lsstcorp.org/secure/Tests.jspa\#/testCase/LVV-T144}{Verify
implementation of Task Specification (DMS-REQ-0305)}\tabularnewline
\protect\hyperlink{lvv-t148---verify-implementation-of-unique-processing-coverage-dms-req-0307}{LVV-T148}
&
\href{https://jira.lsstcorp.org/secure/Tests.jspa\#/testCase/LVV-T148}{Verify
implementation of Unique Processing Coverage
(DMS-REQ-0307)}\tabularnewline
\protect\hyperlink{lvv-t149---verify-implementation-of-catalog-queries-dms-req-0075}{LVV-T149}
&
\href{https://jira.lsstcorp.org/secure/Tests.jspa\#/testCase/LVV-T149}{Verify
implementation of Catalog Queries (DMS-REQ-0075)}\tabularnewline
\protect\hyperlink{lvv-t150---verify-implementation-of-maintain-archive-publicly-accessible-dms-req-0077}{LVV-T150}
&
\href{https://jira.lsstcorp.org/secure/Tests.jspa\#/testCase/LVV-T150}{Verify
implementation of Maintain Archive Publicly Accessible
(DMS-REQ-0077)}\tabularnewline
\protect\hyperlink{lvv-t151---verify-implementation-of-catalog-export-formats-dms-req-0078}{LVV-T151}
&
\href{https://jira.lsstcorp.org/secure/Tests.jspa\#/testCase/LVV-T151}{Verify
implementation of Catalog Export Formats (DMS-REQ-0078)}\tabularnewline
\protect\hyperlink{lvv-t152---verify-implementation-of-keep-historical-alert-archive-dms-req-0094}{LVV-T152}
&
\href{https://jira.lsstcorp.org/secure/Tests.jspa\#/testCase/LVV-T152}{Verify
implementation of Keep Historical Alert Archive
(DMS-REQ-0094)}\tabularnewline
\protect\hyperlink{lvv-t154---verify-implementation-of-raw-data-archiving-reliability-dms-req-0309}{LVV-T154}
&
\href{https://jira.lsstcorp.org/secure/Tests.jspa\#/testCase/LVV-T154}{Verify
implementation of Raw Data Archiving Reliability
(DMS-REQ-0309)}\tabularnewline
\protect\hyperlink{lvv-t155---verify-implementation-of-un-archived-data-product-cache-dms-req-0310}{LVV-T155}
&
\href{https://jira.lsstcorp.org/secure/Tests.jspa\#/testCase/LVV-T155}{Verify
implementation of Un-Archived Data Product Cache
(DMS-REQ-0310)}\tabularnewline
\protect\hyperlink{lvv-t157---verify-implementation-level-1-data-product-access-dms-req-0312}{LVV-T157}
&
\href{https://jira.lsstcorp.org/secure/Tests.jspa\#/testCase/LVV-T157}{Verify
implementation Level 1 Data Product Access
(DMS-REQ-0312)}\tabularnewline
\protect\hyperlink{lvv-t158---verify-implementation-level-1-and-2-catalog-access-dms-req-0313}{LVV-T158}
&
\href{https://jira.lsstcorp.org/secure/Tests.jspa\#/testCase/LVV-T158}{Verify
implementation Level 1 and 2 Catalog Access
(DMS-REQ-0313)}\tabularnewline
\protect\hyperlink{lvv-t159---verify-implementation-of-regenerating-data-products-from-previous-data-releases-dms-req-0336}{LVV-T159}
&
\href{https://jira.lsstcorp.org/secure/Tests.jspa\#/testCase/LVV-T159}{Verify
implementation of Regenerating Data Products from Previous Data Releases
(DMS-REQ-0336)}\tabularnewline
\protect\hyperlink{lvv-t160---verify-implementation-of-providing-a-precovery-service-dms-req-0341}{LVV-T160}
&
\href{https://jira.lsstcorp.org/secure/Tests.jspa\#/testCase/LVV-T160}{Verify
implementation of Providing a Precovery Service
(DMS-REQ-0341)}\tabularnewline
\protect\hyperlink{lvv-t161---verify-implementation-of-logging-of-catalog-queries-dms-req-0345}{LVV-T161}
&
\href{https://jira.lsstcorp.org/secure/Tests.jspa\#/testCase/LVV-T161}{Verify
implementation of Logging of catalog queries
(DMS-REQ-0345)}\tabularnewline
\protect\hyperlink{lvv-t163---verify-implementation-of-data-access-services-dms-req-0364}{LVV-T163}
&
\href{https://jira.lsstcorp.org/secure/Tests.jspa\#/testCase/LVV-T163}{Verify
implementation of Data Access Services (DMS-REQ-0364)}\tabularnewline
\protect\hyperlink{lvv-t164---verify-implementation-of-operations-subsets-dms-req-0365}{LVV-T164}
&
\href{https://jira.lsstcorp.org/secure/Tests.jspa\#/testCase/LVV-T164}{Verify
implementation of Operations Subsets (DMS-REQ-0365)}\tabularnewline
\protect\hyperlink{lvv-t165---verify-implementation-of-subsets-support-dms-req-0366}{LVV-T165}
&
\href{https://jira.lsstcorp.org/secure/Tests.jspa\#/testCase/LVV-T165}{Verify
implementation of Subsets Support (DMS-REQ-0366)}\tabularnewline
\protect\hyperlink{lvv-t168---verify-design-of-data-access-services-allows-evolution-of-the-lsst-data-model-dms-req-0369}{LVV-T168}
&
\href{https://jira.lsstcorp.org/secure/Tests.jspa\#/testCase/LVV-T168}{Verify
design of Data Access Services allows Evolution of the LSST Data Model
(DMS-REQ-0369)}\tabularnewline
\protect\hyperlink{lvv-t169---verify-implementation-of-older-release-behavior-dms-req-0370}{LVV-T169}
&
\href{https://jira.lsstcorp.org/secure/Tests.jspa\#/testCase/LVV-T169}{Verify
implementation of Older Release Behavior (DMS-REQ-0370)}\tabularnewline
\protect\hyperlink{lvv-t170---verify-implementation-of-query-availability-dms-req-0371}{LVV-T170}
&
\href{https://jira.lsstcorp.org/secure/Tests.jspa\#/testCase/LVV-T170}{Verify
implementation of Query Availability (DMS-REQ-0371)}\tabularnewline
\protect\hyperlink{lvv-t171---verify-implementation-of-pipeline-availability-dms-req-0008}{LVV-T171}
&
\href{https://jira.lsstcorp.org/secure/Tests.jspa\#/testCase/LVV-T171}{Verify
implementation of Pipeline Availability (DMS-REQ-0008)}\tabularnewline
\protect\hyperlink{lvv-t172---verify-implementation-of-optimization-of-cost-reliability-and-availability-dms-req-0161}{LVV-T172}
&
\href{https://jira.lsstcorp.org/secure/Tests.jspa\#/testCase/LVV-T172}{Verify
implementation of Optimization of Cost, Reliability and Availability
(DMS-REQ-0161)}\tabularnewline
\protect\hyperlink{lvv-t173---verify-implementation-of-pipeline-throughput-dms-req-0162}{LVV-T173}
&
\href{https://jira.lsstcorp.org/secure/Tests.jspa\#/testCase/LVV-T173}{Verify
implementation of Pipeline Throughput (DMS-REQ-0162)}\tabularnewline
\protect\hyperlink{lvv-t174---verify-implementation-of-re-processing-capacity-dms-req-0163}{LVV-T174}
&
\href{https://jira.lsstcorp.org/secure/Tests.jspa\#/testCase/LVV-T174}{Verify
implementation of Re-processing Capacity (DMS-REQ-0163)}\tabularnewline
\protect\hyperlink{lvv-t180---verify-implementation-of-data-management-unscheduled-downtime-dms-req-0318}{LVV-T180}
&
\href{https://jira.lsstcorp.org/secure/Tests.jspa\#/testCase/LVV-T180}{Verify
implementation of Data Management Unscheduled Downtime
(DMS-REQ-0318)}\tabularnewline
\protect\hyperlink{lvv-t181---verify-implementation-of-summit-facility-data-communications-dms-req-0168}{LVV-T181}
&
\href{https://jira.lsstcorp.org/secure/Tests.jspa\#/testCase/LVV-T181}{Verify
implementation of Summit Facility Data Communications
(DMS-REQ-0168)}\tabularnewline
\protect\hyperlink{lvv-t182---verify-implementation-of-prefer-computing-and-storage-down-dms-req-0170}{LVV-T182}
&
\href{https://jira.lsstcorp.org/secure/Tests.jspa\#/testCase/LVV-T182}{Verify
implementation of Prefer Computing and Storage Down
(DMS-REQ-0170)}\tabularnewline
\protect\hyperlink{lvv-t183---verify-implementation-of-dms-communication-with-ocs-dms-req-0315}{LVV-T183}
&
\href{https://jira.lsstcorp.org/secure/Tests.jspa\#/testCase/LVV-T183}{Verify
implementation of DMS Communication with OCS
(DMS-REQ-0315)}\tabularnewline
\protect\hyperlink{lvv-t184---verify-implementation-of-summit-to-base-network-dms-req-0171}{LVV-T184}
&
\href{https://jira.lsstcorp.org/secure/Tests.jspa\#/testCase/LVV-T184}{Verify
implementation of Summit to Base Network (DMS-REQ-0171)}\tabularnewline
\protect\hyperlink{lvv-t185---verify-implementation-of-summit-to-base-network-availability-dms-req-0172}{LVV-T185}
&
\href{https://jira.lsstcorp.org/secure/Tests.jspa\#/testCase/LVV-T185}{Verify
implementation of Summit to Base Network Availability
(DMS-REQ-0172)}\tabularnewline
\protect\hyperlink{lvv-t186---verify-implementation-of-summit-to-base-network-reliability-dms-req-0173}{LVV-T186}
&
\href{https://jira.lsstcorp.org/secure/Tests.jspa\#/testCase/LVV-T186}{Verify
implementation of Summit to Base Network Reliability
(DMS-REQ-0173)}\tabularnewline
\protect\hyperlink{lvv-t187---verify-implementation-of-summit-to-base-network-secondary-link-dms-req-0174}{LVV-T187}
&
\href{https://jira.lsstcorp.org/secure/Tests.jspa\#/testCase/LVV-T187}{Verify
implementation of Summit to Base Network Secondary Link
(DMS-REQ-0174)}\tabularnewline
\protect\hyperlink{lvv-t188---verify-implementation-of-summit-to-base-network-ownership-and-operation-dms-req-0175}{LVV-T188}
&
\href{https://jira.lsstcorp.org/secure/Tests.jspa\#/testCase/LVV-T188}{Verify
implementation of Summit to Base Network Ownership and Operation
(DMS-REQ-0175)}\tabularnewline
\protect\hyperlink{lvv-t189---verify-implementation-of-base-facility-infrastructure-dms-req-0176}{LVV-T189}
&
\href{https://jira.lsstcorp.org/secure/Tests.jspa\#/testCase/LVV-T189}{Verify
implementation of Base Facility Infrastructure
(DMS-REQ-0176)}\tabularnewline
\protect\hyperlink{lvv-t190---verify-implementation-of-base-facility-co-location-with-existing-facility-dms-req-0178}{LVV-T190}
&
\href{https://jira.lsstcorp.org/secure/Tests.jspa\#/testCase/LVV-T190}{Verify
implementation of Base Facility Co-Location with Existing Facility
(DMS-REQ-0178)}\tabularnewline
\protect\hyperlink{lvv-t191---verify-implementation-of-commissioning-cluster-dms-req-0316}{LVV-T191}
&
\href{https://jira.lsstcorp.org/secure/Tests.jspa\#/testCase/LVV-T191}{Verify
implementation of Commissioning Cluster (DMS-REQ-0316)}\tabularnewline
\protect\hyperlink{lvv-t192---verify-implementation-of-base-wireless-lan-wifi-dms-req-0352}{LVV-T192}
&
\href{https://jira.lsstcorp.org/secure/Tests.jspa\#/testCase/LVV-T192}{Verify
implementation of Base Wireless LAN (WiFi)
(DMS-REQ-0352)}\tabularnewline
\protect\hyperlink{lvv-t193---verify-implementation-of-base-to-archive-network-dms-req-0180}{LVV-T193}
&
\href{https://jira.lsstcorp.org/secure/Tests.jspa\#/testCase/LVV-T193}{Verify
implementation of Base to Archive Network (DMS-REQ-0180)}\tabularnewline
\protect\hyperlink{lvv-t194---verify-implementation-of-base-to-archive-network-availability-dms-req-0181}{LVV-T194}
&
\href{https://jira.lsstcorp.org/secure/Tests.jspa\#/testCase/LVV-T194}{Verify
implementation of Base to Archive Network Availability
(DMS-REQ-0181)}\tabularnewline
\protect\hyperlink{lvv-t195---verify-implementation-of-base-to-archive-network-reliability-dms-req-0182}{LVV-T195}
&
\href{https://jira.lsstcorp.org/secure/Tests.jspa\#/testCase/LVV-T195}{Verify
implementation of Base to Archive Network Reliability
(DMS-REQ-0182)}\tabularnewline
\protect\hyperlink{lvv-t196---verify-implementation-of-base-to-archive-network-secondary-link-dms-req-0183}{LVV-T196}
&
\href{https://jira.lsstcorp.org/secure/Tests.jspa\#/testCase/LVV-T196}{Verify
implementation of Base to Archive Network Secondary Link
(DMS-REQ-0183)}\tabularnewline
\protect\hyperlink{lvv-t197---verify-implementation-of-archive-center-dms-req-0185}{LVV-T197}
&
\href{https://jira.lsstcorp.org/secure/Tests.jspa\#/testCase/LVV-T197}{Verify
implementation of Archive Center (DMS-REQ-0185)}\tabularnewline
\protect\hyperlink{lvv-t198---verify-implementation-of--archive-center-disaster-recovery-dms-req-0186}{LVV-T198}
&
\href{https://jira.lsstcorp.org/secure/Tests.jspa\#/testCase/LVV-T198}{Verify
implementation of Archive Center Disaster Recovery
(DMS-REQ-0186)}\tabularnewline
\protect\hyperlink{lvv-t199---verify-implementation-of-archive-center-co-location-with-existing-facility-dms-req-0187}{LVV-T199}
&
\href{https://jira.lsstcorp.org/secure/Tests.jspa\#/testCase/LVV-T199}{Verify
implementation of Archive Center Co-Location with Existing Facility
(DMS-REQ-0187)}\tabularnewline
\protect\hyperlink{lvv-t200---verify-implementation-of-archive-to-data-access-center-network-dms-req-0188}{LVV-T200}
&
\href{https://jira.lsstcorp.org/secure/Tests.jspa\#/testCase/LVV-T200}{Verify
implementation of Archive to Data Access Center Network
(DMS-REQ-0188)}\tabularnewline
\protect\hyperlink{lvv-t201---verify-implementation-of-archive-to-data-access-center-network-availability-dms-req-0189}{LVV-T201}
&
\href{https://jira.lsstcorp.org/secure/Tests.jspa\#/testCase/LVV-T201}{Verify
implementation of Archive to Data Access Center Network Availability
(DMS-REQ-0189)}\tabularnewline
\protect\hyperlink{lvv-t202---verify-implementation-of-archive-to-data-access-center-network-reliability-dms-req-0190}{LVV-T202}
&
\href{https://jira.lsstcorp.org/secure/Tests.jspa\#/testCase/LVV-T202}{Verify
implementation of Archive to Data Access Center Network Reliability
(DMS-REQ-0190)}\tabularnewline
\protect\hyperlink{lvv-t204---verify-implementation-of-access-to-catalogs-for-external-level-3-processing-dms-req-0122}{LVV-T204}
&
\href{https://jira.lsstcorp.org/secure/Tests.jspa\#/testCase/LVV-T204}{Verify
implementation of Access to catalogs for external Level 3 processing
(DMS-REQ-0122)}\tabularnewline
\protect\hyperlink{lvv-t205---verify-implementation-of-access-to-input-catalogs-for-dac-based-level-3-processing-dms-req-0123}{LVV-T205}
&
\href{https://jira.lsstcorp.org/secure/Tests.jspa\#/testCase/LVV-T205}{Verify
implementation of Access to input catalogs for DAC-based Level 3
processing (DMS-REQ-0123)}\tabularnewline
\protect\hyperlink{lvv-t207---verify-implementation-of-access-to-images-for-external-level-3-processing-dms-req-0126}{LVV-T207}
&
\href{https://jira.lsstcorp.org/secure/Tests.jspa\#/testCase/LVV-T207}{Verify
implementation of Access to images for external Level 3 processing
(DMS-REQ-0126)}\tabularnewline
\protect\hyperlink{lvv-t208---verify-implementation-of-access-to-input-images-for-dac-based-level-3-processing-dms-req-0127}{LVV-T208}
&
\href{https://jira.lsstcorp.org/secure/Tests.jspa\#/testCase/LVV-T208}{Verify
implementation of Access to input images for DAC-based Level 3
processing (DMS-REQ-0127)}\tabularnewline
\protect\hyperlink{lvv-t212---verify-implementation-of-no-limit-on-data-access-centers-dms-req-0197}{LVV-T212}
&
\href{https://jira.lsstcorp.org/secure/Tests.jspa\#/testCase/LVV-T212}{Verify
implementation of No Limit on Data Access Centers
(DMS-REQ-0197)}\tabularnewline
\bottomrule
\end{longtable}

\section{Test Cases}\label{test-cases}

\hypertarget{lvv-t120---verify-implementation-of-software-framework-for-level-3-catalog-processing-dms-req-0125}{\subsection{\texorpdfstring{\href{https://jira.lsstcorp.org/secure/Tests.jspa\#/testCase/LVV-T120}{LVV-T120}
- Verify implementation of Software framework for Level 3 catalog
processing
(DMS-REQ-0125)}{LVV-T120 - Verify implementation of Software framework for Level 3 catalog processing (DMS-REQ-0125)}}\label{lvv-t120---verify-implementation-of-software-framework-for-level-3-catalog-processing-dms-req-0125}}

\begin{longtable}[]{@{}llllll@{}}
\toprule
Version & Status & Priority & Verification Type & Critical Event &
Owner\tabularnewline
\midrule
\endhead
1 & Draft & Normal & Test & False & Colin Slater\tabularnewline
\bottomrule
\end{longtable}

\subsubsection{Test Items}\label{test-items}

Verify that user-driven Level 3 processing can be consistently applied
to all records in a catalog.

\subsubsection{Requirements}\label{requirements}

\begin{itemize}
\tightlist
\item
  \href{https://jira.lsstcorp.org/browse/LVV-53}{LVV-53} -
  DMS-REQ-0125-V-01: Software framework for Level 3 catalog processing
\end{itemize}

\subsubsection{Test Script}\label{test-script}

\textbf{Step 1}\\
Execute representative processing on DR in PDAC, observe recognition of
and recovery from failures\\[2\baselineskip]

\hypertarget{lvv-t121---verify-implementation-of-software-framework-for-level-3-image-processing-dms-req-0128}{\subsection{\texorpdfstring{\href{https://jira.lsstcorp.org/secure/Tests.jspa\#/testCase/LVV-T121}{LVV-T121}
- Verify implementation of Software framework for Level 3 image
processing
(DMS-REQ-0128)}{LVV-T121 - Verify implementation of Software framework for Level 3 image processing (DMS-REQ-0128)}}\label{lvv-t121---verify-implementation-of-software-framework-for-level-3-image-processing-dms-req-0128}}

\begin{longtable}[]{@{}llllll@{}}
\toprule
Version & Status & Priority & Verification Type & Critical Event &
Owner\tabularnewline
\midrule
\endhead
1 & Draft & Normal & Test & False & Colin Slater\tabularnewline
\bottomrule
\end{longtable}

\subsubsection{Test Items}\label{test-items-1}

Verify that user-specified Level 3 processing can be applied to the
desired set of images.

\subsubsection{Requirements}\label{requirements-1}

\begin{itemize}
\tightlist
\item
  \href{https://jira.lsstcorp.org/browse/LVV-56}{LVV-56} -
  DMS-REQ-0128-V-01: Software framework for Level 3 image processing
\end{itemize}

\subsubsection{Test Script}\label{test-script-1}

\textbf{Step 1}\\
Execute representative processing on DR in PDAC, observe recognition of
and recovery from failures\\[2\baselineskip]

\hypertarget{lvv-t123---verify-implementation-of-access-controls-of-level-3-data-products-dms-req-0340}{\subsection{\texorpdfstring{\href{https://jira.lsstcorp.org/secure/Tests.jspa\#/testCase/LVV-T123}{LVV-T123}
- Verify implementation of Access Controls of Level 3 Data Products
(DMS-REQ-0340)}{LVV-T123 - Verify implementation of Access Controls of Level 3 Data Products (DMS-REQ-0340)}}\label{lvv-t123---verify-implementation-of-access-controls-of-level-3-data-products-dms-req-0340}}

\begin{longtable}[]{@{}llllll@{}}
\toprule
Version & Status & Priority & Verification Type & Critical Event &
Owner\tabularnewline
\midrule
\endhead
1 & Draft & Normal & Test & False & Robert Gruendl\tabularnewline
\bottomrule
\end{longtable}

\subsubsection{Test Items}\label{test-items-2}

This test touches upon the interface between the following areas: IT
Security, Identity Management, LSP Portal, and Parallel Distributed
Database. ~The purpose is to show that access to user generated data
products (previously Level 3) can have a variety of access restrictions
varying from single-user, a list, a named group, or open access.

\subsubsection{Requirements}\label{requirements-2}

\begin{itemize}
\tightlist
\item
  \href{https://jira.lsstcorp.org/browse/LVV-171}{LVV-171} -
  DMS-REQ-0340-V-01: Access Controls of Level 3 Data Products
\end{itemize}

\subsubsection{Precondition}\label{precondition}

\subsubsection{Test Script}\label{test-script-2}

\textbf{Step 1}\\
Configure representative access controls in PDAC, observe proper
restrictions\\[2\baselineskip]

\hypertarget{lvv-t125---verify-implementation-of-simulated-data-dms-req-0009}{\subsection{\texorpdfstring{\href{https://jira.lsstcorp.org/secure/Tests.jspa\#/testCase/LVV-T125}{LVV-T125}
- Verify implementation of Simulated Data
(DMS-REQ-0009)}{LVV-T125 - Verify implementation of Simulated Data (DMS-REQ-0009)}}\label{lvv-t125---verify-implementation-of-simulated-data-dms-req-0009}}

\begin{longtable}[]{@{}llllll@{}}
\toprule
Version & Status & Priority & Verification Type & Critical Event &
Owner\tabularnewline
\midrule
\endhead
1 & Draft & Normal & Test & False & Robert Lupton\tabularnewline
\bottomrule
\end{longtable}

\subsubsection{Test Items}\label{test-items-3}

Verify that the DMS can inject simulated data into data products for
testing.

\subsubsection{Requirements}\label{requirements-3}

\begin{itemize}
\tightlist
\item
  \href{https://jira.lsstcorp.org/browse/LVV-6}{LVV-6} -
  DMS-REQ-0009-V-01: Simulated Data
\end{itemize}

\subsubsection{Test Script}\label{test-script-3}

\textbf{Step 1}\\
Delegate to AP and DRP\\[2\baselineskip]

\subsection{\texorpdfstring{\href{https://jira.lsstcorp.org/secure/Tests.jspa\#/testCase/LVV-T126}{LVV-T126}
- Verify implementation Image Differencing
(DMS-REQ-0032)}{LVV-T126 - Verify implementation Image Differencing (DMS-REQ-0032)}}\label{lvv-t126---verify-implementation-image-differencing-dms-req-0032}

\begin{longtable}[]{@{}llllll@{}}
\toprule
Version & Status & Priority & Verification Type & Critical Event &
Owner\tabularnewline
\midrule
\endhead
1 & Draft & Normal & Test & False & Eric Bellm\tabularnewline
\bottomrule
\end{longtable}

\subsubsection{Test Items}\label{test-items-4}

Verify that the DMS can performance image differencing from single
exposures and coadds.

\subsubsection{Requirements}\label{requirements-4}

\begin{itemize}
\tightlist
\item
  \href{https://jira.lsstcorp.org/browse/LVV-14}{LVV-14} -
  DMS-REQ-0032-V-01: Image Differencing
\end{itemize}

\subsubsection{Test Script}\label{test-script-4}

\textbf{Step 1}\\
Delegate to AP and DRP\\[2\baselineskip]

\hypertarget{lvv-t127---verify-implementation-of-provide-source-detection-software-dms-req-0033}{\subsection{\texorpdfstring{\href{https://jira.lsstcorp.org/secure/Tests.jspa\#/testCase/LVV-T127}{LVV-T127}
- Verify implementation of Provide Source Detection Software
(DMS-REQ-0033)}{LVV-T127 - Verify implementation of Provide Source Detection Software (DMS-REQ-0033)}}\label{lvv-t127---verify-implementation-of-provide-source-detection-software-dms-req-0033}}

\begin{longtable}[]{@{}llllll@{}}
\toprule
Version & Status & Priority & Verification Type & Critical Event &
Owner\tabularnewline
\midrule
\endhead
1 & Draft & Normal & Test & False & Robert Lupton\tabularnewline
\bottomrule
\end{longtable}

\subsubsection{Test Items}\label{test-items-5}

Alert Production, Data Release Production, Science Algorithms~

\subsubsection{Requirements}\label{requirements-5}

\begin{itemize}
\tightlist
\item
  \href{https://jira.lsstcorp.org/browse/LVV-15}{LVV-15} -
  DMS-REQ-0033-V-01: Provide Source Detection Software
\end{itemize}

\subsubsection{Test Script}\label{test-script-5}

\textbf{Step 1}\\
Delegate to AP and DRP\\[2\baselineskip]

\hypertarget{lvv-t128---verify-implementation-provide-astrometric-model-dms-req-0042}{\subsection{\texorpdfstring{\href{https://jira.lsstcorp.org/secure/Tests.jspa\#/testCase/LVV-T128}{LVV-T128}
- Verify implementation Provide Astrometric Model
(DMS-REQ-0042)}{LVV-T128 - Verify implementation Provide Astrometric Model (DMS-REQ-0042)}}\label{lvv-t128---verify-implementation-provide-astrometric-model-dms-req-0042}}

\begin{longtable}[]{@{}llllll@{}}
\toprule
Version & Status & Priority & Verification Type & Critical Event &
Owner\tabularnewline
\midrule
\endhead
1 & Draft & Normal & Test & False & Colin Slater\tabularnewline
\bottomrule
\end{longtable}

\subsubsection{Test Items}\label{test-items-6}

Verify that an astrometric model is available for Objects and
DIAObjects.

\subsubsection{Requirements}\label{requirements-6}

\begin{itemize}
\tightlist
\item
  \href{https://jira.lsstcorp.org/browse/LVV-17}{LVV-17} -
  DMS-REQ-0042-V-01: Provide Astrometric Model
\end{itemize}

\subsubsection{Test Script}\label{test-script-6}

\textbf{Step 1}\\
Delegate to AP and DRP\\[2\baselineskip]

\hypertarget{lvv-t129---verify-implementation-of-provide-calibrated-photometry-dms-req-0043}{\subsection{\texorpdfstring{\href{https://jira.lsstcorp.org/secure/Tests.jspa\#/testCase/LVV-T129}{LVV-T129}
- Verify implementation of Provide Calibrated Photometry
(DMS-REQ-0043)}{LVV-T129 - Verify implementation of Provide Calibrated Photometry (DMS-REQ-0043)}}\label{lvv-t129---verify-implementation-of-provide-calibrated-photometry-dms-req-0043}}

\begin{longtable}[]{@{}llllll@{}}
\toprule
Version & Status & Priority & Verification Type & Critical Event &
Owner\tabularnewline
\midrule
\endhead
1 & Draft & Normal & Test & False & Robert Lupton\tabularnewline
\bottomrule
\end{longtable}

\subsubsection{Test Items}\label{test-items-7}

Verify that the DMS provides photometry calibrated in AB for all
measured objects and sources.

\subsubsection{Requirements}\label{requirements-7}

\begin{itemize}
\tightlist
\item
  \href{https://jira.lsstcorp.org/browse/LVV-18}{LVV-18} -
  DMS-REQ-0043-V-01: Provide Calibrated Photometry
\end{itemize}

\subsubsection{Test Script}\label{test-script-7}

\textbf{Step 1}\\
Delegate to AP and DRP\\[2\baselineskip]

\hypertarget{lvv-t130---verify-implementation-of-enable-a-range-of-shape-measurement-approaches-dms-req-0052}{\subsection{\texorpdfstring{\href{https://jira.lsstcorp.org/secure/Tests.jspa\#/testCase/LVV-T130}{LVV-T130}
- Verify implementation of Enable a Range of Shape Measurement
Approaches
(DMS-REQ-0052)}{LVV-T130 - Verify implementation of Enable a Range of Shape Measurement Approaches (DMS-REQ-0052)}}\label{lvv-t130---verify-implementation-of-enable-a-range-of-shape-measurement-approaches-dms-req-0052}}

\begin{longtable}[]{@{}llllll@{}}
\toprule
Version & Status & Priority & Verification Type & Critical Event &
Owner\tabularnewline
\midrule
\endhead
1 & Draft & Normal & Test & False & Colin Slater\tabularnewline
\bottomrule
\end{longtable}

\subsubsection{Test Items}\label{test-items-8}

Verify that multiple shape measurement algorithms can be used.

\subsubsection{Requirements}\label{requirements-8}

\begin{itemize}
\tightlist
\item
  \href{https://jira.lsstcorp.org/browse/LVV-21}{LVV-21} -
  DMS-REQ-0052-V-01: Enable a Range of Shape Measurement Approaches
\end{itemize}

\subsubsection{Test Script}\label{test-script-8}

\textbf{Step 1}\\
Delegate to AP and DRP\\[2\baselineskip]

\hypertarget{lvv-t132---verify-implementation-of-pre-cursor-and-real-data-dms-req-0296}{\subsection{\texorpdfstring{\href{https://jira.lsstcorp.org/secure/Tests.jspa\#/testCase/LVV-T132}{LVV-T132}
- Verify implementation of Pre-cursor, and Real Data
(DMS-REQ-0296)}{LVV-T132 - Verify implementation of Pre-cursor, and Real Data (DMS-REQ-0296)}}\label{lvv-t132---verify-implementation-of-pre-cursor-and-real-data-dms-req-0296}}

\begin{longtable}[]{@{}llllll@{}}
\toprule
Version & Status & Priority & Verification Type & Critical Event &
Owner\tabularnewline
\midrule
\endhead
1 & Draft & Normal & Test & False & Robert Gruendl\tabularnewline
\bottomrule
\end{longtable}

\subsubsection{Test Items}\label{test-items-9}

Demonstrate that pixel-oriented data from astronomical imaging cameras
(precursor or otherwise) can be processed using LSST Science Algorithms
and organized for access through the Data Butler Access Client. ~

\subsubsection{Requirements}\label{requirements-9}

\begin{itemize}
\tightlist
\item
  \href{https://jira.lsstcorp.org/browse/LVV-127}{LVV-127} -
  DMS-REQ-0296-V-01: Pre-cursor, and Real Data
\end{itemize}

\subsubsection{Test Script}\label{test-script-9}

\textbf{Step 1}\\
Execute AP and DRP on non-LSST data\\[2\baselineskip]

\hypertarget{lvv-t133---verify-implementation-of-provide-beam-projector-coordinate-calculation-software-dms-req-0351}{\subsection{\texorpdfstring{\href{https://jira.lsstcorp.org/secure/Tests.jspa\#/testCase/LVV-T133}{LVV-T133}
- Verify implementation of Provide Beam Projector Coordinate Calculation
Software
(DMS-REQ-0351)}{LVV-T133 - Verify implementation of Provide Beam Projector Coordinate Calculation Software (DMS-REQ-0351)}}\label{lvv-t133---verify-implementation-of-provide-beam-projector-coordinate-calculation-software-dms-req-0351}}

\begin{longtable}[]{@{}llllll@{}}
\toprule
Version & Status & Priority & Verification Type & Critical Event &
Owner\tabularnewline
\midrule
\endhead
1 & Draft & Normal & Test & False & Robert Lupton\tabularnewline
\bottomrule
\end{longtable}

\subsubsection{Test Items}\label{test-items-10}

Science Primitives~

\subsubsection{Requirements}\label{requirements-10}

\begin{itemize}
\tightlist
\item
  \href{https://jira.lsstcorp.org/browse/LVV-182}{LVV-182} -
  DMS-REQ-0351-V-01: Provide Beam Projector Coordinate Calculation
  Software
\end{itemize}

\subsubsection{Test Script}\label{test-script-10}

\textbf{Step 1}\\
Delegate to CPP\\[2\baselineskip]

\hypertarget{lvv-t134---verify-implementation-of-provide-image-access-services-dms-req-0065}{\subsection{\texorpdfstring{\href{https://jira.lsstcorp.org/secure/Tests.jspa\#/testCase/LVV-T134}{LVV-T134}
- Verify implementation of Provide Image Access Services
(DMS-REQ-0065)}{LVV-T134 - Verify implementation of Provide Image Access Services (DMS-REQ-0065)}}\label{lvv-t134---verify-implementation-of-provide-image-access-services-dms-req-0065}}

\begin{longtable}[]{@{}llllll@{}}
\toprule
Version & Status & Priority & Verification Type & Critical Event &
Owner\tabularnewline
\midrule
\endhead
1 & Draft & Normal & Test & False & Gregory
Dubois-Felsmann\tabularnewline
\bottomrule
\end{longtable}

\subsubsection{Test Items}\label{test-items-11}

Verify that images can be identified and that images and image cut-outs
can be retrieved using the network interfaces - primarily IVOA
standards-based - and Python APIs provided for image access by science
users.

\subsubsection{Requirements}\label{requirements-11}

\begin{itemize}
\tightlist
\item
  \href{https://jira.lsstcorp.org/browse/LVV-27}{LVV-27} -
  DMS-REQ-0065-V-01: Provide Image Access Services
\end{itemize}

\subsubsection{Precondition}\label{precondition-1}

Testing requires the establishment of running services such as SIAv2 and
SODA to which the tests can be applied.

\subsubsection{Test Script}\label{test-script-11}

\textbf{Step 1}\\
Delegate to LSP\\[2\baselineskip]

\hypertarget{lvv-t135---verify-implementation-of-provide-data-access-services-dms-req-0155}{\subsection{\texorpdfstring{\href{https://jira.lsstcorp.org/secure/Tests.jspa\#/testCase/LVV-T135}{LVV-T135}
- Verify implementation of Provide Data Access Services
(DMS-REQ-0155)}{LVV-T135 - Verify implementation of Provide Data Access Services (DMS-REQ-0155)}}\label{lvv-t135---verify-implementation-of-provide-data-access-services-dms-req-0155}}

\begin{longtable}[]{@{}llllll@{}}
\toprule
Version & Status & Priority & Verification Type & Critical Event &
Owner\tabularnewline
\midrule
\endhead
1 & Draft & Normal & Test & False & Robert Gruendl\tabularnewline
\bottomrule
\end{longtable}

\subsubsection{Test Items}\label{test-items-12}

This is a composite requirement in the SysML model.

\subsubsection{Requirements}\label{requirements-12}

\begin{itemize}
\tightlist
\item
  \href{https://jira.lsstcorp.org/browse/LVV-60}{LVV-60} -
  DMS-REQ-0155-V-01: Provide Data Access Services
\end{itemize}

\subsubsection{Test Script}\label{test-script-12}

\textbf{Step 1}\\
Delegate to LSP\\[2\baselineskip]

\hypertarget{lvv-t136---verify-implementation-of-data-product-and-raw-data-access-dms-req-0298}{\subsection{\texorpdfstring{\href{https://jira.lsstcorp.org/secure/Tests.jspa\#/testCase/LVV-T136}{LVV-T136}
- Verify implementation of Data Product and Raw Data Access
(DMS-REQ-0298)}{LVV-T136 - Verify implementation of Data Product and Raw Data Access (DMS-REQ-0298)}}\label{lvv-t136---verify-implementation-of-data-product-and-raw-data-access-dms-req-0298}}

\begin{longtable}[]{@{}llllll@{}}
\toprule
Version & Status & Priority & Verification Type & Critical Event &
Owner\tabularnewline
\midrule
\endhead
1 & Draft & Normal & Test & False & Colin Slater\tabularnewline
\bottomrule
\end{longtable}

\subsubsection{Test Items}\label{test-items-13}

Verify that available data products can be listed and retrieved.

\subsubsection{Requirements}\label{requirements-13}

\begin{itemize}
\tightlist
\item
  \href{https://jira.lsstcorp.org/browse/LVV-129}{LVV-129} -
  DMS-REQ-0298-V-01: Data Product and Raw Data Access
\end{itemize}

\subsubsection{Test Script}\label{test-script-13}

\textbf{Step 1}\\
Delegate to LSP\\[2\baselineskip]

\hypertarget{lvv-t137---verify-implementation-of-data-product-ingest-dms-req-0299}{\subsection{\texorpdfstring{\href{https://jira.lsstcorp.org/secure/Tests.jspa\#/testCase/LVV-T137}{LVV-T137}
- Verify implementation of Data Product Ingest
(DMS-REQ-0299)}{LVV-T137 - Verify implementation of Data Product Ingest (DMS-REQ-0299)}}\label{lvv-t137---verify-implementation-of-data-product-ingest-dms-req-0299}}

\begin{longtable}[]{@{}llllll@{}}
\toprule
Version & Status & Priority & Verification Type & Critical Event &
Owner\tabularnewline
\midrule
\endhead
1 & Draft & Normal & Test & False & Colin Slater\tabularnewline
\bottomrule
\end{longtable}

\subsubsection{Test Items}\label{test-items-14}

Verify that data products can be ingested.

\subsubsection{Requirements}\label{requirements-14}

\begin{itemize}
\tightlist
\item
  \href{https://jira.lsstcorp.org/browse/LVV-130}{LVV-130} -
  DMS-REQ-0299-V-01: Data Product Ingest
\end{itemize}

\subsubsection{Test Script}\label{test-script-14}

\textbf{Step 1}\\
Delegate to DBB\\[2\baselineskip]

\hypertarget{lvv-t143---verify-implementation-of-provide-pipeline-construction-services-dms-req-0158}{\subsection{\texorpdfstring{\href{https://jira.lsstcorp.org/secure/Tests.jspa\#/testCase/LVV-T143}{LVV-T143}
- Verify implementation of Provide Pipeline Construction Services
(DMS-REQ-0158)}{LVV-T143 - Verify implementation of Provide Pipeline Construction Services (DMS-REQ-0158)}}\label{lvv-t143---verify-implementation-of-provide-pipeline-construction-services-dms-req-0158}}

\begin{longtable}[]{@{}llllll@{}}
\toprule
Version & Status & Priority & Verification Type & Critical Event &
Owner\tabularnewline
\midrule
\endhead
1 & Draft & Normal & Test & False & Robert Lupton\tabularnewline
\bottomrule
\end{longtable}

\subsubsection{Test Items}\label{test-items-15}

This is a composite requirement in the SysML model

\subsubsection{Requirements}\label{requirements-15}

\begin{itemize}
\tightlist
\item
  \href{https://jira.lsstcorp.org/browse/LVV-62}{LVV-62} -
  DMS-REQ-0158-V-01: Provide Pipeline Construction Services
\end{itemize}

\subsubsection{Test Script}\label{test-script-15}

\textbf{Step 1}\\
Delegate to Middleware\\[2\baselineskip]

\hypertarget{lvv-t144---verify-implementation-of-task-specification-dms-req-0305}{\subsection{\texorpdfstring{\href{https://jira.lsstcorp.org/secure/Tests.jspa\#/testCase/LVV-T144}{LVV-T144}
- Verify implementation of Task Specification
(DMS-REQ-0305)}{LVV-T144 - Verify implementation of Task Specification (DMS-REQ-0305)}}\label{lvv-t144---verify-implementation-of-task-specification-dms-req-0305}}

\begin{longtable}[]{@{}llllll@{}}
\toprule
Version & Status & Priority & Verification Type & Critical Event &
Owner\tabularnewline
\midrule
\endhead
1 & Draft & Normal & Test & False & Kian-Tat Lim\tabularnewline
\bottomrule
\end{longtable}

\subsubsection{Test Items}\label{test-items-16}

Verify that the DMS provides the ability to define a new or modified
pipeline task without recompilation.

\subsubsection{Requirements}\label{requirements-16}

\begin{itemize}
\tightlist
\item
  \href{https://jira.lsstcorp.org/browse/LVV-136}{LVV-136} -
  DMS-REQ-0305-V-01: Task Specification
\end{itemize}

\subsubsection{Test Script}\label{test-script-16}

\textbf{Step 1}\\
Inspect software architecture. ~Verify that there exists Tasks that can
be run and configured without
re-complication.\\[2\baselineskip]\textbf{Step 2}\\
Verify that an example science algorithm can be run through one of these
Tasks.~ Three examples from different areas: source measurement, image
subtraction, and photometric-redshift estimation.\\[2\baselineskip]

\hypertarget{lvv-t148---verify-implementation-of-unique-processing-coverage-dms-req-0307}{\subsection{\texorpdfstring{\href{https://jira.lsstcorp.org/secure/Tests.jspa\#/testCase/LVV-T148}{LVV-T148}
- Verify implementation of Unique Processing Coverage
(DMS-REQ-0307)}{LVV-T148 - Verify implementation of Unique Processing Coverage (DMS-REQ-0307)}}\label{lvv-t148---verify-implementation-of-unique-processing-coverage-dms-req-0307}}

\begin{longtable}[]{@{}llllll@{}}
\toprule
Version & Status & Priority & Verification Type & Critical Event &
Owner\tabularnewline
\midrule
\endhead
1 & Draft & Normal & Test & False & Colin Slater\tabularnewline
\bottomrule
\end{longtable}

\subsubsection{Test Items}\label{test-items-17}

Verify that a user-specified criterion can be used to process each
record in a table exactly once.

\subsubsection{Requirements}\label{requirements-17}

\begin{itemize}
\tightlist
\item
  \href{https://jira.lsstcorp.org/browse/LVV-138}{LVV-138} -
  DMS-REQ-0307-V-01: Unique Processing Coverage
\end{itemize}

\subsubsection{Test Script}\label{test-script-17}

\textbf{Step 1}\\
Execute representative processing, observe lack of duplicates or missing
rows even in the presence of failures\\[2\baselineskip]

\hypertarget{lvv-t149---verify-implementation-of-catalog-queries-dms-req-0075}{\subsection{\texorpdfstring{\href{https://jira.lsstcorp.org/secure/Tests.jspa\#/testCase/LVV-T149}{LVV-T149}
- Verify implementation of Catalog Queries
(DMS-REQ-0075)}{LVV-T149 - Verify implementation of Catalog Queries (DMS-REQ-0075)}}\label{lvv-t149---verify-implementation-of-catalog-queries-dms-req-0075}}

\begin{longtable}[]{@{}llllll@{}}
\toprule
Version & Status & Priority & Verification Type & Critical Event &
Owner\tabularnewline
\midrule
\endhead
1 & Draft & Normal & Test & False & Colin Slater\tabularnewline
\bottomrule
\end{longtable}

\subsubsection{Test Items}\label{test-items-18}

Verify that SQL can be used to query catalogs.

\subsubsection{Requirements}\label{requirements-18}

\begin{itemize}
\tightlist
\item
  \href{https://jira.lsstcorp.org/browse/LVV-33}{LVV-33} -
  DMS-REQ-0075-V-01: Catalog Queries
\end{itemize}

\subsubsection{Test Script}\label{test-script-18}

\textbf{Step 1}\\
Delegate to LSP\\[2\baselineskip]

\hypertarget{lvv-t150---verify-implementation-of-maintain-archive-publicly-accessible-dms-req-0077}{\subsection{\texorpdfstring{\href{https://jira.lsstcorp.org/secure/Tests.jspa\#/testCase/LVV-T150}{LVV-T150}
- Verify implementation of Maintain Archive Publicly Accessible
(DMS-REQ-0077)}{LVV-T150 - Verify implementation of Maintain Archive Publicly Accessible (DMS-REQ-0077)}}\label{lvv-t150---verify-implementation-of-maintain-archive-publicly-accessible-dms-req-0077}}

\begin{longtable}[]{@{}llllll@{}}
\toprule
Version & Status & Priority & Verification Type & Critical Event &
Owner\tabularnewline
\midrule
\endhead
1 & Draft & Normal & Test & False & Colin Slater\tabularnewline
\bottomrule
\end{longtable}

\subsubsection{Test Items}\label{test-items-19}

Verify that prior data releases remain accessible.

\subsubsection{Requirements}\label{requirements-19}

\begin{itemize}
\tightlist
\item
  \href{https://jira.lsstcorp.org/browse/LVV-34}{LVV-34} -
  DMS-REQ-0077-V-01: Maintain Archive Publicly Accessible
\end{itemize}

\subsubsection{Test Script}\label{test-script-19}

\textbf{Step 1}\\
Observe access to prior DR on tape\\[2\baselineskip]

\hypertarget{lvv-t151---verify-implementation-of-catalog-export-formats-dms-req-0078}{\subsection{\texorpdfstring{\href{https://jira.lsstcorp.org/secure/Tests.jspa\#/testCase/LVV-T151}{LVV-T151}
- Verify implementation of Catalog Export Formats
(DMS-REQ-0078)}{LVV-T151 - Verify implementation of Catalog Export Formats (DMS-REQ-0078)}}\label{lvv-t151---verify-implementation-of-catalog-export-formats-dms-req-0078}}

\begin{longtable}[]{@{}llllll@{}}
\toprule
Version & Status & Priority & Verification Type & Critical Event &
Owner\tabularnewline
\midrule
\endhead
1 & Draft & Normal & Test & False & Colin Slater\tabularnewline
\bottomrule
\end{longtable}

\subsubsection{Test Items}\label{test-items-20}

Verify that catalog data is exportable in a variety of
community-standard formats.

\subsubsection{Requirements}\label{requirements-20}

\begin{itemize}
\tightlist
\item
  \href{https://jira.lsstcorp.org/browse/LVV-35}{LVV-35} -
  DMS-REQ-0078-V-01: Catalog Export Formats
\end{itemize}

\subsubsection{Test Script}\label{test-script-20}

\textbf{Step 1}\\
Delegate to LSP\\[2\baselineskip]

\hypertarget{lvv-t152---verify-implementation-of-keep-historical-alert-archive-dms-req-0094}{\subsection{\texorpdfstring{\href{https://jira.lsstcorp.org/secure/Tests.jspa\#/testCase/LVV-T152}{LVV-T152}
- Verify implementation of Keep Historical Alert Archive
(DMS-REQ-0094)}{LVV-T152 - Verify implementation of Keep Historical Alert Archive (DMS-REQ-0094)}}\label{lvv-t152---verify-implementation-of-keep-historical-alert-archive-dms-req-0094}}

\begin{longtable}[]{@{}llllll@{}}
\toprule
Version & Status & Priority & Verification Type & Critical Event &
Owner\tabularnewline
\midrule
\endhead
1 & Draft & Normal & Test & False & Eric Bellm\tabularnewline
\bottomrule
\end{longtable}

\subsubsection{Test Items}\label{test-items-21}

Verify that the DMS preserves and makes accessible an Alert Archive for
reference and for false alert analyses

\subsubsection{Requirements}\label{requirements-21}

\begin{itemize}
\tightlist
\item
  \href{https://jira.lsstcorp.org/browse/LVV-37}{LVV-37} -
  DMS-REQ-0094-V-01: Keep Historical Alert Archive
\end{itemize}

\subsubsection{Test Script}\label{test-script-21}

\textbf{Step 1}\\
Simulated alert stream, load Alert DB, observe access to Alert
DB\\[2\baselineskip]

\hypertarget{lvv-t154---verify-implementation-of-raw-data-archiving-reliability-dms-req-0309}{\subsection{\texorpdfstring{\href{https://jira.lsstcorp.org/secure/Tests.jspa\#/testCase/LVV-T154}{LVV-T154}
- Verify implementation of Raw Data Archiving Reliability
(DMS-REQ-0309)}{LVV-T154 - Verify implementation of Raw Data Archiving Reliability (DMS-REQ-0309)}}\label{lvv-t154---verify-implementation-of-raw-data-archiving-reliability-dms-req-0309}}

\begin{longtable}[]{@{}llllll@{}}
\toprule
Version & Status & Priority & Verification Type & Critical Event &
Owner\tabularnewline
\midrule
\endhead
1 & Draft & Normal & Test & False & Colin Slater\tabularnewline
\bottomrule
\end{longtable}

\subsubsection{Test Items}\label{test-items-22}

Verify that raw images are reliably archived.

\subsubsection{Requirements}\label{requirements-22}

\begin{itemize}
\tightlist
\item
  \href{https://jira.lsstcorp.org/browse/LVV-140}{LVV-140} -
  DMS-REQ-0309-V-01: Raw Data Archiving Reliability
\end{itemize}

\subsubsection{Test Script}\label{test-script-22}

\textbf{Step 1}\\
Analyze sources of loss or corruption after mitigation to compute
estimated reliability\\[2\baselineskip]

\hypertarget{lvv-t155---verify-implementation-of-un-archived-data-product-cache-dms-req-0310}{\subsection{\texorpdfstring{\href{https://jira.lsstcorp.org/secure/Tests.jspa\#/testCase/LVV-T155}{LVV-T155}
- Verify implementation of Un-Archived Data Product Cache
(DMS-REQ-0310)}{LVV-T155 - Verify implementation of Un-Archived Data Product Cache (DMS-REQ-0310)}}\label{lvv-t155---verify-implementation-of-un-archived-data-product-cache-dms-req-0310}}

\begin{longtable}[]{@{}llllll@{}}
\toprule
Version & Status & Priority & Verification Type & Critical Event &
Owner\tabularnewline
\midrule
\endhead
1 & Draft & Normal & Test & False & Robert Gruendl\tabularnewline
\bottomrule
\end{longtable}

\subsubsection{Test Items}\label{test-items-23}

Demonstrate that the DMS provides low-latency storage for at least
I1CacheLifetime (30 days) to keep prompt processing pre-covery images on
hand.

\subsubsection{Requirements}\label{requirements-23}

\begin{itemize}
\tightlist
\item
  \href{https://jira.lsstcorp.org/browse/LVV-141}{LVV-141} -
  DMS-REQ-0310-V-01: Un-Archived Data Product Cache
\end{itemize}

\subsubsection{Test Script}\label{test-script-23}

\textbf{Step 1}\\
Delegate to DBB\\[2\baselineskip]

\hypertarget{lvv-t157---verify-implementation-level-1-data-product-access-dms-req-0312}{\subsection{\texorpdfstring{\href{https://jira.lsstcorp.org/secure/Tests.jspa\#/testCase/LVV-T157}{LVV-T157}
- Verify implementation Level 1 Data Product Access
(DMS-REQ-0312)}{LVV-T157 - Verify implementation Level 1 Data Product Access (DMS-REQ-0312)}}\label{lvv-t157---verify-implementation-level-1-data-product-access-dms-req-0312}}

\begin{longtable}[]{@{}llllll@{}}
\toprule
Version & Status & Priority & Verification Type & Critical Event &
Owner\tabularnewline
\midrule
\endhead
1 & Draft & Normal & Test & False & Colin Slater\tabularnewline
\bottomrule
\end{longtable}

\subsubsection{Test Items}\label{test-items-24}

Verify that Level 1 Data Products are accessible by science users.

\subsubsection{Requirements}\label{requirements-24}

\begin{itemize}
\tightlist
\item
  \href{https://jira.lsstcorp.org/browse/LVV-143}{LVV-143} -
  DMS-REQ-0312-V-01: Level 1 Data Product Access
\end{itemize}

\subsubsection{Test Script}\label{test-script-24}

\textbf{Step 1}\\
Delegate to LSP\\[2\baselineskip]

\hypertarget{lvv-t158---verify-implementation-level-1-and-2-catalog-access-dms-req-0313}{\subsection{\texorpdfstring{\href{https://jira.lsstcorp.org/secure/Tests.jspa\#/testCase/LVV-T158}{LVV-T158}
- Verify implementation Level 1 and 2 Catalog Access
(DMS-REQ-0313)}{LVV-T158 - Verify implementation Level 1 and 2 Catalog Access (DMS-REQ-0313)}}\label{lvv-t158---verify-implementation-level-1-and-2-catalog-access-dms-req-0313}}

\begin{longtable}[]{@{}llllll@{}}
\toprule
Version & Status & Priority & Verification Type & Critical Event &
Owner\tabularnewline
\midrule
\endhead
1 & Draft & Normal & Test & False & Colin Slater\tabularnewline
\bottomrule
\end{longtable}

\subsubsection{Test Items}\label{test-items-25}

Verify that Data Release Products are accessible by science users.

\subsubsection{Requirements}\label{requirements-25}

\begin{itemize}
\tightlist
\item
  \href{https://jira.lsstcorp.org/browse/LVV-144}{LVV-144} -
  DMS-REQ-0313-V-01: Level 1 \& 2 Catalog Access
\end{itemize}

\subsubsection{Test Script}\label{test-script-25}

\textbf{Step 1}\\
Delegate to LSP\\[2\baselineskip]

\hypertarget{lvv-t159---verify-implementation-of-regenerating-data-products-from-previous-data-releases-dms-req-0336}{\subsection{\texorpdfstring{\href{https://jira.lsstcorp.org/secure/Tests.jspa\#/testCase/LVV-T159}{LVV-T159}
- Verify implementation of Regenerating Data Products from Previous Data
Releases
(DMS-REQ-0336)}{LVV-T159 - Verify implementation of Regenerating Data Products from Previous Data Releases (DMS-REQ-0336)}}\label{lvv-t159---verify-implementation-of-regenerating-data-products-from-previous-data-releases-dms-req-0336}}

\begin{longtable}[]{@{}llllll@{}}
\toprule
Version & Status & Priority & Verification Type & Critical Event &
Owner\tabularnewline
\midrule
\endhead
1 & Draft & Normal & Test & False & Simon Krughoff\tabularnewline
\bottomrule
\end{longtable}

\subsubsection{Test Items}\label{test-items-26}

Show that un-archived data products from previous data releases can be
generated using through the LSST Science Platform.

\subsubsection{Requirements}\label{requirements-26}

\begin{itemize}
\tightlist
\item
  \href{https://jira.lsstcorp.org/browse/LVV-167}{LVV-167} -
  DMS-REQ-0336-V-01: Regenerating Data Products from Previous Data
  Releases
\end{itemize}

\subsubsection{Test Script}\label{test-script-26}

\textbf{Step 1}\\
Delegate to LSP\\[2\baselineskip]

\hypertarget{lvv-t160---verify-implementation-of-providing-a-precovery-service-dms-req-0341}{\subsection{\texorpdfstring{\href{https://jira.lsstcorp.org/secure/Tests.jspa\#/testCase/LVV-T160}{LVV-T160}
- Verify implementation of Providing a Precovery Service
(DMS-REQ-0341)}{LVV-T160 - Verify implementation of Providing a Precovery Service (DMS-REQ-0341)}}\label{lvv-t160---verify-implementation-of-providing-a-precovery-service-dms-req-0341}}

\begin{longtable}[]{@{}llllll@{}}
\toprule
Version & Status & Priority & Verification Type & Critical Event &
Owner\tabularnewline
\midrule
\endhead
1 & Draft & Normal & Test & False & Gregory
Dubois-Felsmann\tabularnewline
\bottomrule
\end{longtable}

\subsubsection{Test Items}\label{test-items-27}

Verify that a technical capability to perform user-directed precovery
analyses on difference images exists and that it is exposed through the
LSST Science Platform. ~Verified by testing against precursor
datasets.\\
(Involves: LSP Portal, MOPS and Forced Photometry)

\subsubsection{Requirements}\label{requirements-27}

\begin{itemize}
\tightlist
\item
  \href{https://jira.lsstcorp.org/browse/LVV-172}{LVV-172} -
  DMS-REQ-0341-V-01: Providing a Precovery Service
\end{itemize}

\subsubsection{Precondition}\label{precondition-2}

\begin{enumerate}
\tightlist
\item
  DECam HiTS data could be an appropriate set for this activity.
\item
  Precovery pipelines for follow-on to alert processing must exist and
  be made available as a containerized version within the Science
  Platform.
\item
  Determine limitations over which general precovery is supported. ~I
  would suggest that precovery services be limited to current (or last
  two) DRP campaigns with the possible addition of including non-DRP
  products to encompass observations over the preceding year (does this
  then require means to re-generate PVIs from Alert Production in
  addition to DRP?)
\item
  Could re-use elements of
  \href{https://jira.lsstcorp.org/secure/Tests.jspa\#/testCase/LVV-T80}{LVV-T80}
  where quasars are used to test faint object detection.
\end{enumerate}

\subsubsection{Test Script}\label{test-script-27}

\textbf{Step 1}\\
Run Precovery within follow-on Alert Production (i.e. daily
post-processing on 30 day store).\\[2\baselineskip]\textbf{Step 2}\\
Within Science Platform, initiate request to perform precovery for a
list of sources over same period (and longer). ~Include among the
sources for precovery quasars from
\href{https://jira.lsstcorp.org/secure/Tests.jspa\#/testCase/LVV-T80}{LVV-T80}.~\\[2\baselineskip]\textbf{Step
3}\\
Examine the results. ~Compare the results for the period where there is
overlap with precovery run\ldots{} and quasar photometry with those from
\href{https://jira.lsstcorp.org/secure/Tests.jspa\#/testCase/LVV-T80}{LVV-T80}
to verify user service performs as production
services.\\[2\baselineskip]

\hypertarget{lvv-t161---verify-implementation-of-logging-of-catalog-queries-dms-req-0345}{\subsection{\texorpdfstring{\href{https://jira.lsstcorp.org/secure/Tests.jspa\#/testCase/LVV-T161}{LVV-T161}
- Verify implementation of Logging of catalog queries
(DMS-REQ-0345)}{LVV-T161 - Verify implementation of Logging of catalog queries (DMS-REQ-0345)}}\label{lvv-t161---verify-implementation-of-logging-of-catalog-queries-dms-req-0345}}

\begin{longtable}[]{@{}llllll@{}}
\toprule
Version & Status & Priority & Verification Type & Critical Event &
Owner\tabularnewline
\midrule
\endhead
1 & Draft & Normal & Test & False & Robert Gruendl\tabularnewline
\bottomrule
\end{longtable}

\subsubsection{Test Items}\label{test-items-28}

Demonstrate logging of queries of LSST databases. ~Logged queries are
globally available to DB administrators but otherwise private excepting
the user that made the query.

\subsubsection{Requirements}\label{requirements-28}

\begin{itemize}
\tightlist
\item
  \href{https://jira.lsstcorp.org/browse/LVV-176}{LVV-176} -
  DMS-REQ-0345-V-01: Logging of catalog queries
\end{itemize}

\subsubsection{Test Script}\label{test-script-28}

\textbf{Step 1}\\
Delegate to LSP\\[2\baselineskip]

\hypertarget{lvv-t163---verify-implementation-of-data-access-services-dms-req-0364}{\subsection{\texorpdfstring{\href{https://jira.lsstcorp.org/secure/Tests.jspa\#/testCase/LVV-T163}{LVV-T163}
- Verify implementation of Data Access Services
(DMS-REQ-0364)}{LVV-T163 - Verify implementation of Data Access Services (DMS-REQ-0364)}}\label{lvv-t163---verify-implementation-of-data-access-services-dms-req-0364}}

\begin{longtable}[]{@{}llllll@{}}
\toprule
Version & Status & Priority & Verification Type & Critical Event &
Owner\tabularnewline
\midrule
\endhead
1 & Draft & Normal & Test & False & Robert Gruendl\tabularnewline
\bottomrule
\end{longtable}

\subsubsection{Test Items}\label{test-items-29}

Demonstrate that Data Access Services are capable of scaling to serve
data from nDRTot (11) data releases over a surveyYears (10) year survey.

\subsubsection{Requirements}\label{requirements-29}

\begin{itemize}
\tightlist
\item
  \href{https://jira.lsstcorp.org/browse/LVV-190}{LVV-190} -
  DMS-REQ-0364-V-01: Data Access Services
\end{itemize}

\subsubsection{Test Script}\label{test-script-29}

\textbf{Step 1}\\
Delegate to LSP\\[2\baselineskip]

\hypertarget{lvv-t164---verify-implementation-of-operations-subsets-dms-req-0365}{\subsection{\texorpdfstring{\href{https://jira.lsstcorp.org/secure/Tests.jspa\#/testCase/LVV-T164}{LVV-T164}
- Verify implementation of Operations Subsets
(DMS-REQ-0365)}{LVV-T164 - Verify implementation of Operations Subsets (DMS-REQ-0365)}}\label{lvv-t164---verify-implementation-of-operations-subsets-dms-req-0365}}

\begin{longtable}[]{@{}llllll@{}}
\toprule
Version & Status & Priority & Verification Type & Critical Event &
Owner\tabularnewline
\midrule
\endhead
1 & Draft & Normal & Test & False & Robert Gruendl\tabularnewline
\bottomrule
\end{longtable}

\subsubsection{Test Items}\label{test-items-30}

Demonstrate that Data Access Services are designed such that subsets of
a Data Release may be retained and served (made available) after a Data
Release has been superseded. ~ (Data Backbone, Managed Database, LSP
Portal, LSP JupyterLab, LSP Web APIs, Parallel Distributed Database)

\subsubsection{Requirements}\label{requirements-30}

\begin{itemize}
\tightlist
\item
  \href{https://jira.lsstcorp.org/browse/LVV-191}{LVV-191} -
  DMS-REQ-0365-V-01: Operations Subsets
\end{itemize}

\subsubsection{Test Script}\label{test-script-30}

\textbf{Step 1}\\
Delegate to LSP\\[2\baselineskip]

\hypertarget{lvv-t165---verify-implementation-of-subsets-support-dms-req-0366}{\subsection{\texorpdfstring{\href{https://jira.lsstcorp.org/secure/Tests.jspa\#/testCase/LVV-T165}{LVV-T165}
- Verify implementation of Subsets Support
(DMS-REQ-0366)}{LVV-T165 - Verify implementation of Subsets Support (DMS-REQ-0366)}}\label{lvv-t165---verify-implementation-of-subsets-support-dms-req-0366}}

\begin{longtable}[]{@{}llllll@{}}
\toprule
Version & Status & Priority & Verification Type & Critical Event &
Owner\tabularnewline
\midrule
\endhead
1 & Draft & Normal & Test & False & Robert Lupton\tabularnewline
\bottomrule
\end{longtable}

\subsubsection{Test Items}\label{test-items-31}

Verify that the DMS can provide designated subsets of previous Data
Releases.

\subsubsection{Requirements}\label{requirements-31}

\begin{itemize}
\tightlist
\item
  \href{https://jira.lsstcorp.org/browse/LVV-192}{LVV-192} -
  DMS-REQ-0366-V-01: Subsets Support
\end{itemize}

\subsubsection{Test Script}\label{test-script-31}

\textbf{Step 1}\\
Delegate to LSP\\[2\baselineskip]

\hypertarget{lvv-t168---verify-design-of-data-access-services-allows-evolution-of-the-lsst-data-model-dms-req-0369}{\subsection{\texorpdfstring{\href{https://jira.lsstcorp.org/secure/Tests.jspa\#/testCase/LVV-T168}{LVV-T168}
- Verify design of Data Access Services allows Evolution of the LSST
Data Model
(DMS-REQ-0369)}{LVV-T168 - Verify design of Data Access Services allows Evolution of the LSST Data Model (DMS-REQ-0369)}}\label{lvv-t168---verify-design-of-data-access-services-allows-evolution-of-the-lsst-data-model-dms-req-0369}}

\begin{longtable}[]{@{}llllll@{}}
\toprule
Version & Status & Priority & Verification Type & Critical Event &
Owner\tabularnewline
\midrule
\endhead
1 & Draft & Normal & Test & False & Robert Gruendl\tabularnewline
\bottomrule
\end{longtable}

\subsubsection{Test Items}\label{test-items-32}

Verify that the design of the Data Access Services are able to
accommodate changes/evolution of the LSST data model from one release to
another.

\subsubsection{Requirements}\label{requirements-32}

\begin{itemize}
\tightlist
\item
  \href{https://jira.lsstcorp.org/browse/LVV-195}{LVV-195} -
  DMS-REQ-0369-V-01: Evolution
\end{itemize}

\subsubsection{Test Script}\label{test-script-32}

\textbf{Step 1}\\
Delegate to LSP\\[2\baselineskip]

\hypertarget{lvv-t169---verify-implementation-of-older-release-behavior-dms-req-0370}{\subsection{\texorpdfstring{\href{https://jira.lsstcorp.org/secure/Tests.jspa\#/testCase/LVV-T169}{LVV-T169}
- Verify implementation of Older Release Behavior
(DMS-REQ-0370)}{LVV-T169 - Verify implementation of Older Release Behavior (DMS-REQ-0370)}}\label{lvv-t169---verify-implementation-of-older-release-behavior-dms-req-0370}}

\begin{longtable}[]{@{}llllll@{}}
\toprule
Version & Status & Priority & Verification Type & Critical Event &
Owner\tabularnewline
\midrule
\endhead
1 & Draft & Normal & Test & False & Gregory
Dubois-Felsmann\tabularnewline
\bottomrule
\end{longtable}

\subsubsection{Test Items}\label{test-items-33}

Verify that the components of the data access system are technically
capable of handling data releases beyond the two for which full services
are required. ~DMS-REQ-0364 requires that up to 11 be supported.
~Verified by inspection, i.e., by determination that the system design
and implementation contain the necessary features to support this number
of releases, and by direct test in a synthetic test environment with
multiple releases.\\
(Involves: Data Backbone, Managed Database, LSP Portal, LSP JupyterLab,
LSP Web APIs, Parallel Distributed Database)

\subsubsection{Requirements}\label{requirements-33}

\begin{itemize}
\tightlist
\item
  \href{https://jira.lsstcorp.org/browse/LVV-196}{LVV-196} -
  DMS-REQ-0370-V-01: Older Release Behavior
\end{itemize}

\subsubsection{Test Script}\label{test-script-33}

\textbf{Step 1}\\
Delegate to LSP\\[2\baselineskip]

\hypertarget{lvv-t170---verify-implementation-of-query-availability-dms-req-0371}{\subsection{\texorpdfstring{\href{https://jira.lsstcorp.org/secure/Tests.jspa\#/testCase/LVV-T170}{LVV-T170}
- Verify implementation of Query Availability
(DMS-REQ-0371)}{LVV-T170 - Verify implementation of Query Availability (DMS-REQ-0371)}}\label{lvv-t170---verify-implementation-of-query-availability-dms-req-0371}}

\begin{longtable}[]{@{}llllll@{}}
\toprule
Version & Status & Priority & Verification Type & Critical Event &
Owner\tabularnewline
\midrule
\endhead
1 & Draft & Normal & Test & False & Colin Slater\tabularnewline
\bottomrule
\end{longtable}

\subsubsection{Test Items}\label{test-items-34}

Verify that queries continue to be successfully executable over time.

\subsubsection{Requirements}\label{requirements-34}

\begin{itemize}
\tightlist
\item
  \href{https://jira.lsstcorp.org/browse/LVV-197}{LVV-197} -
  DMS-REQ-0371-V-01: Query Availability
\end{itemize}

\subsubsection{Test Script}\label{test-script-34}

\textbf{Step 1}\\
Delegate to LSP\\[2\baselineskip]

\hypertarget{lvv-t171---verify-implementation-of-pipeline-availability-dms-req-0008}{\subsection{\texorpdfstring{\href{https://jira.lsstcorp.org/secure/Tests.jspa\#/testCase/LVV-T171}{LVV-T171}
- Verify implementation of Pipeline Availability
(DMS-REQ-0008)}{LVV-T171 - Verify implementation of Pipeline Availability (DMS-REQ-0008)}}\label{lvv-t171---verify-implementation-of-pipeline-availability-dms-req-0008}}

\begin{longtable}[]{@{}llllll@{}}
\toprule
Version & Status & Priority & Verification Type & Critical Event &
Owner\tabularnewline
\midrule
\endhead
1 & Draft & Normal & Test & False & Robert Gruendl\tabularnewline
\bottomrule
\end{longtable}

\subsubsection{Test Items}\label{test-items-35}

Demonstrate that Data Management System pipelines are available for use
without disruptions of greater than productionMaxDowntime (24 hours). ~
This requires a regimented change control process and testing
infrastructure for all pipelines and their underlying software services,
and regimented management and monitoring of compute and networking
resources. ~The list of services covered by this test include: Image and
EFD Archiving, Prompt Processing, OCS Driven Batch, Telemetry Gateway,
Alert Distribution, Alert Filtering, Batch Production, Data Backbone,
Compute/Storage/LAN, Inter-Site Networks, and Service Management and
Monitoring.

\subsubsection{Requirements}\label{requirements-35}

\begin{itemize}
\tightlist
\item
  \href{https://jira.lsstcorp.org/browse/LVV-5}{LVV-5} -
  DMS-REQ-0008-V-01: Pipeline Availability
\end{itemize}

\subsubsection{Test Script}\label{test-script-35}

\textbf{Step 1}\\
Analyze sources of downtime after mitigation to compute estimated
reliability; observe unscheduled downtime of developer, integration, and
pre-production systems\\[2\baselineskip]

\hypertarget{lvv-t172---verify-implementation-of-optimization-of-cost-reliability-and-availability-dms-req-0161}{\subsection{\texorpdfstring{\href{https://jira.lsstcorp.org/secure/Tests.jspa\#/testCase/LVV-T172}{LVV-T172}
- Verify implementation of Optimization of Cost, Reliability and
Availability
(DMS-REQ-0161)}{LVV-T172 - Verify implementation of Optimization of Cost, Reliability and Availability (DMS-REQ-0161)}}\label{lvv-t172---verify-implementation-of-optimization-of-cost-reliability-and-availability-dms-req-0161}}

\begin{longtable}[]{@{}llllll@{}}
\toprule
Version & Status & Priority & Verification Type & Critical Event &
Owner\tabularnewline
\midrule
\endhead
1 & Draft & Normal & Test & False & Robert Gruendl\tabularnewline
\bottomrule
\end{longtable}

\subsubsection{Test Items}\label{test-items-36}

In matters of cost, system reliability (functioning properly at a given
time) has precedence over system availability (ability to use the system
at a given time). ~ The optimization may be outside the realm of direct
testing as it is more of a system provisioning guideline but on its face
it demands that the Data Management System include failure reporting,
regimented change control, acceptance testing, maintenance and
monitoring.

\subsubsection{Requirements}\label{requirements-36}

\begin{itemize}
\tightlist
\item
  \href{https://jira.lsstcorp.org/browse/LVV-64}{LVV-64} -
  DMS-REQ-0161-V-01: Optimization of Cost, Reliability and Availability
  in Order
\end{itemize}

\subsubsection{Test Script}\label{test-script-36}

\textbf{Step 1}\\
Analyze resource management policy\\[2\baselineskip]

\hypertarget{lvv-t173---verify-implementation-of-pipeline-throughput-dms-req-0162}{\subsection{\texorpdfstring{\href{https://jira.lsstcorp.org/secure/Tests.jspa\#/testCase/LVV-T173}{LVV-T173}
- Verify implementation of Pipeline Throughput
(DMS-REQ-0162)}{LVV-T173 - Verify implementation of Pipeline Throughput (DMS-REQ-0162)}}\label{lvv-t173---verify-implementation-of-pipeline-throughput-dms-req-0162}}

\begin{longtable}[]{@{}llllll@{}}
\toprule
Version & Status & Priority & Verification Type & Critical Event &
Owner\tabularnewline
\midrule
\endhead
1 & Draft & Normal & Test & False & Robert Gruendl\tabularnewline
\bottomrule
\end{longtable}

\subsubsection{Test Items}\label{test-items-37}

Demonstrate that the Alert Production Pipeline is capable of processing
nRawExpNightMax (2800) science exposures within a (24-nightDurationMax)
12 hour period and issue alerts in offline batch mode.~

\subsubsection{Requirements}\label{requirements-37}

\begin{itemize}
\tightlist
\item
  \href{https://jira.lsstcorp.org/browse/LVV-65}{LVV-65} -
  DMS-REQ-0162-V-01: Pipeline Throughput
\end{itemize}

\subsubsection{Test Script}\label{test-script-37}

\textbf{Step 1}\\
Execute single-day operations rehearsal, observe data products generated
in time\\[2\baselineskip]

\hypertarget{lvv-t174---verify-implementation-of-re-processing-capacity-dms-req-0163}{\subsection{\texorpdfstring{\href{https://jira.lsstcorp.org/secure/Tests.jspa\#/testCase/LVV-T174}{LVV-T174}
- Verify implementation of Re-processing Capacity
(DMS-REQ-0163)}{LVV-T174 - Verify implementation of Re-processing Capacity (DMS-REQ-0163)}}\label{lvv-t174---verify-implementation-of-re-processing-capacity-dms-req-0163}}

\begin{longtable}[]{@{}llllll@{}}
\toprule
Version & Status & Priority & Verification Type & Critical Event &
Owner\tabularnewline
\midrule
\endhead
1 & Draft & Normal & Test & False & Robert Gruendl\tabularnewline
\bottomrule
\end{longtable}

\subsubsection{Test Items}\label{test-items-38}

Verify that the DMS has sufficient processing, storage, and network to
reprocess all data within ``drProcessingPeriod'' (1 year) while
maintaining full Prompt Processing capability.

\subsubsection{Requirements}\label{requirements-38}

\begin{itemize}
\tightlist
\item
  \href{https://jira.lsstcorp.org/browse/LVV-66}{LVV-66} -
  DMS-REQ-0163-V-01: Re-processing Capacity
\end{itemize}

\subsubsection{Test Script}\label{test-script-38}

\textbf{Step 1}\\
\hspace*{0.333em}Analyze sizing model; execute DRP, observe
scaling\\[2\baselineskip]

\hypertarget{lvv-t180---verify-implementation-of-data-management-unscheduled-downtime-dms-req-0318}{\subsection{\texorpdfstring{\href{https://jira.lsstcorp.org/secure/Tests.jspa\#/testCase/LVV-T180}{LVV-T180}
- Verify implementation of Data Management Unscheduled Downtime
(DMS-REQ-0318)}{LVV-T180 - Verify implementation of Data Management Unscheduled Downtime (DMS-REQ-0318)}}\label{lvv-t180---verify-implementation-of-data-management-unscheduled-downtime-dms-req-0318}}

\begin{longtable}[]{@{}llllll@{}}
\toprule
Version & Status & Priority & Verification Type & Critical Event &
Owner\tabularnewline
\midrule
\endhead
1 & Draft & Normal & Test & False & Robert Gruendl\tabularnewline
\bottomrule
\end{longtable}

\subsubsection{Test Items}\label{test-items-39}

This applies only to downtime that would prevent the collection of
survey data. ~Verification means that analysis has occurred to identify
likely hardware failures that would prevent survey operations and that
mitigations that minimize the downtime to less than DMDowntime (1
day/year) are in place. ~Known systems that fall in this category
include: Image and EFD Archiving, Observatory Operations Data, Telemetry
Gateway, Data Backbone, Managed Database, Inter-Site Networks, and
Service Management and Monitoring.~

\subsubsection{Requirements}\label{requirements-39}

\begin{itemize}
\tightlist
\item
  \href{https://jira.lsstcorp.org/browse/LVV-149}{LVV-149} -
  DMS-REQ-0318-V-01: Data Management Unscheduled Downtime
\end{itemize}

\subsubsection{Test Script}\label{test-script-39}

\textbf{Step 1}\\
Analyze likely hardware failures with mitigations to compute estimated
unplanned downtime\\[2\baselineskip]

\hypertarget{lvv-t181---verify-implementation-of-summit-facility-data-communications-dms-req-0168}{\subsection{\texorpdfstring{\href{https://jira.lsstcorp.org/secure/Tests.jspa\#/testCase/LVV-T181}{LVV-T181}
- Verify implementation of Summit Facility Data Communications
(DMS-REQ-0168)}{LVV-T181 - Verify implementation of Summit Facility Data Communications (DMS-REQ-0168)}}\label{lvv-t181---verify-implementation-of-summit-facility-data-communications-dms-req-0168}}

\begin{longtable}[]{@{}llllll@{}}
\toprule
Version & Status & Priority & Verification Type & Critical Event &
Owner\tabularnewline
\midrule
\endhead
1 & Draft & Normal & Test & False & Robert Gruendl\tabularnewline
\bottomrule
\end{longtable}

\subsubsection{Test Items}\label{test-items-40}

Demonstrate data acquisition, archiving and transfer from summit to
base, along with monitoring systems exist and perform adequately.

\subsubsection{Requirements}\label{requirements-40}

\begin{itemize}
\tightlist
\item
  \href{https://jira.lsstcorp.org/browse/LVV-71}{LVV-71} -
  DMS-REQ-0168-V-01: Summit Facility Data Communications
\end{itemize}

\subsubsection{Test Script}\label{test-script-40}

\textbf{Step 1}\\
Delegate to Networks\\[2\baselineskip]

\hypertarget{lvv-t182---verify-implementation-of-prefer-computing-and-storage-down-dms-req-0170}{\subsection{\texorpdfstring{\href{https://jira.lsstcorp.org/secure/Tests.jspa\#/testCase/LVV-T182}{LVV-T182}
- Verify implementation of Prefer Computing and Storage Down
(DMS-REQ-0170)}{LVV-T182 - Verify implementation of Prefer Computing and Storage Down (DMS-REQ-0170)}}\label{lvv-t182---verify-implementation-of-prefer-computing-and-storage-down-dms-req-0170}}

\begin{longtable}[]{@{}llllll@{}}
\toprule
Version & Status & Priority & Verification Type & Critical Event &
Owner\tabularnewline
\midrule
\endhead
1 & Draft & Normal & Test & False & Robert Gruendl\tabularnewline
\bottomrule
\end{longtable}

\subsubsection{Test Items}\label{test-items-41}

Only build compute or storage facilities at the summit that are
justified by operational need or to prevent loss of data during
networking downtimes.

\subsubsection{Requirements}\label{requirements-41}

\begin{itemize}
\tightlist
\item
  \href{https://jira.lsstcorp.org/browse/LVV-72}{LVV-72} -
  DMS-REQ-0170-V-01: Prefer Computing and Storage Down
\end{itemize}

\subsubsection{Test Script}\label{test-script-41}

\textbf{Step 1}\\
Analyze design\\[2\baselineskip]

\hypertarget{lvv-t183---verify-implementation-of-dms-communication-with-ocs-dms-req-0315}{\subsection{\texorpdfstring{\href{https://jira.lsstcorp.org/secure/Tests.jspa\#/testCase/LVV-T183}{LVV-T183}
- Verify implementation of DMS Communication with OCS
(DMS-REQ-0315)}{LVV-T183 - Verify implementation of DMS Communication with OCS (DMS-REQ-0315)}}\label{lvv-t183---verify-implementation-of-dms-communication-with-ocs-dms-req-0315}}

\begin{longtable}[]{@{}llllll@{}}
\toprule
Version & Status & Priority & Verification Type & Critical Event &
Owner\tabularnewline
\midrule
\endhead
1 & Draft & Normal & Test & False & Gregory
Dubois-Felsmann\tabularnewline
\bottomrule
\end{longtable}

\subsubsection{Test Items}\label{test-items-42}

Verify that the DMS at the Base Facility can receive commands from the
OCS and send command responses, events, and telemetry back. ~Verified by
Early Integration activities and during AuxTel commissioning.

\subsubsection{Requirements}\label{requirements-42}

\begin{itemize}
\tightlist
\item
  \href{https://jira.lsstcorp.org/browse/LVV-146}{LVV-146} -
  DMS-REQ-0315-V-01: DMS Communication with OCS
\end{itemize}

\subsubsection{Test Script}\label{test-script-42}

\textbf{Step 1}\\
Delegate to IIP\\[2\baselineskip]

\hypertarget{lvv-t184---verify-implementation-of-summit-to-base-network-dms-req-0171}{\subsection{\texorpdfstring{\href{https://jira.lsstcorp.org/secure/Tests.jspa\#/testCase/LVV-T184}{LVV-T184}
- Verify implementation of Summit to Base Network
(DMS-REQ-0171)}{LVV-T184 - Verify implementation of Summit to Base Network (DMS-REQ-0171)}}\label{lvv-t184---verify-implementation-of-summit-to-base-network-dms-req-0171}}

\begin{longtable}[]{@{}llllll@{}}
\toprule
Version & Status & Priority & Verification Type & Critical Event &
Owner\tabularnewline
\midrule
\endhead
1 & Draft & Normal & Test & False & Robert Gruendl\tabularnewline
\bottomrule
\end{longtable}

\subsubsection{Test Items}\label{test-items-43}

Monitor transfer time of crosstalk corrected images and all related
metadata from summit to base and verify that time per exposure is less
than summToBaseMaxTransferTime (2 seconds).

\subsubsection{Requirements}\label{requirements-43}

\begin{itemize}
\tightlist
\item
  \href{https://jira.lsstcorp.org/browse/LVV-73}{LVV-73} -
  DMS-REQ-0171-V-01: Summit to Base Network
\end{itemize}

\subsubsection{Test Script}\label{test-script-43}

\textbf{Step 1}\\
Delegate to Networks\\[2\baselineskip]

\hypertarget{lvv-t185---verify-implementation-of-summit-to-base-network-availability-dms-req-0172}{\subsection{\texorpdfstring{\href{https://jira.lsstcorp.org/secure/Tests.jspa\#/testCase/LVV-T185}{LVV-T185}
- Verify implementation of Summit to Base Network Availability
(DMS-REQ-0172)}{LVV-T185 - Verify implementation of Summit to Base Network Availability (DMS-REQ-0172)}}\label{lvv-t185---verify-implementation-of-summit-to-base-network-availability-dms-req-0172}}

\begin{longtable}[]{@{}llllll@{}}
\toprule
Version & Status & Priority & Verification Type & Critical Event &
Owner\tabularnewline
\midrule
\endhead
1 & Draft & Normal & Test & False & Robert Gruendl\tabularnewline
\bottomrule
\end{longtable}

\subsubsection{Test Items}\label{test-items-44}

Monitor summit to base networking and verify that the mean time between
failures is less than summToBaseNetMTBF (90 days) over 1 year.

\subsubsection{Requirements}\label{requirements-44}

\begin{itemize}
\tightlist
\item
  \href{https://jira.lsstcorp.org/browse/LVV-74}{LVV-74} -
  DMS-REQ-0172-V-01: Summit to Base Network Availability
\end{itemize}

\subsubsection{Test Script}\label{test-script-44}

\textbf{Step 1}\\
Delegate to Networks\\[2\baselineskip]

\hypertarget{lvv-t186---verify-implementation-of-summit-to-base-network-reliability-dms-req-0173}{\subsection{\texorpdfstring{\href{https://jira.lsstcorp.org/secure/Tests.jspa\#/testCase/LVV-T186}{LVV-T186}
- Verify implementation of Summit to Base Network Reliability
(DMS-REQ-0173)}{LVV-T186 - Verify implementation of Summit to Base Network Reliability (DMS-REQ-0173)}}\label{lvv-t186---verify-implementation-of-summit-to-base-network-reliability-dms-req-0173}}

\begin{longtable}[]{@{}llllll@{}}
\toprule
Version & Status & Priority & Verification Type & Critical Event &
Owner\tabularnewline
\midrule
\endhead
1 & Draft & Normal & Test & False & Robert Gruendl\tabularnewline
\bottomrule
\end{longtable}

\subsubsection{Test Items}\label{test-items-45}

Monitor Summit to Base networking and verify that the mean time to
repair is less than summToBaseNetMTTR (24 hours) over a 1-year period.

\subsubsection{Requirements}\label{requirements-45}

\begin{itemize}
\tightlist
\item
  \href{https://jira.lsstcorp.org/browse/LVV-75}{LVV-75} -
  DMS-REQ-0173-V-01: Summit to Base Network Reliability
\end{itemize}

\subsubsection{Test Script}\label{test-script-45}

\textbf{Step 1}\\
Delegate to Networks\\[2\baselineskip]

\hypertarget{lvv-t187---verify-implementation-of-summit-to-base-network-secondary-link-dms-req-0174}{\subsection{\texorpdfstring{\href{https://jira.lsstcorp.org/secure/Tests.jspa\#/testCase/LVV-T187}{LVV-T187}
- Verify implementation of Summit to Base Network Secondary Link
(DMS-REQ-0174)}{LVV-T187 - Verify implementation of Summit to Base Network Secondary Link (DMS-REQ-0174)}}\label{lvv-t187---verify-implementation-of-summit-to-base-network-secondary-link-dms-req-0174}}

\begin{longtable}[]{@{}llllll@{}}
\toprule
Version & Status & Priority & Verification Type & Critical Event &
Owner\tabularnewline
\midrule
\endhead
1 & Draft & Normal & Test & False & Robert Gruendl\tabularnewline
\bottomrule
\end{longtable}

\subsubsection{Test Items}\label{test-items-46}

A secondary transfer method (redundant fiber network, microwave link, or
transportable medium) between Summit and Base capable of transferring 1
night of raw data (nCalibExpDay + nRawExpNightMax = 450 + 2800 = ~3250
exposures) within summToBaseNet2TransMax (72 hours). ~

\subsubsection{Requirements}\label{requirements-46}

\begin{itemize}
\tightlist
\item
  \href{https://jira.lsstcorp.org/browse/LVV-76}{LVV-76} -
  DMS-REQ-0174-V-01: Summit to Base Network Secondary Link
\end{itemize}

\subsubsection{Test Script}\label{test-script-46}

\textbf{Step 1}\\
Delegate to Networks\\[2\baselineskip]

\hypertarget{lvv-t188---verify-implementation-of-summit-to-base-network-ownership-and-operation-dms-req-0175}{\subsection{\texorpdfstring{\href{https://jira.lsstcorp.org/secure/Tests.jspa\#/testCase/LVV-T188}{LVV-T188}
- Verify implementation of Summit to Base Network Ownership and
Operation
(DMS-REQ-0175)}{LVV-T188 - Verify implementation of Summit to Base Network Ownership and Operation (DMS-REQ-0175)}}\label{lvv-t188---verify-implementation-of-summit-to-base-network-ownership-and-operation-dms-req-0175}}

\begin{longtable}[]{@{}llllll@{}}
\toprule
Version & Status & Priority & Verification Type & Critical Event &
Owner\tabularnewline
\midrule
\endhead
1 & Draft & Normal & Test & False & Robert Gruendl\tabularnewline
\bottomrule
\end{longtable}

\subsubsection{Test Items}\label{test-items-47}

Verify that the Summit to Base communications link is owned and operated
by LSST and/or the operations entity.

\subsubsection{Requirements}\label{requirements-47}

\begin{itemize}
\tightlist
\item
  \href{https://jira.lsstcorp.org/browse/LVV-77}{LVV-77} -
  DMS-REQ-0175-V-01: Summit to Base Network Ownership and Operation
\end{itemize}

\subsubsection{Test Script}\label{test-script-47}

\textbf{Step 1}\\
Delegate to Networks\\[2\baselineskip]

\hypertarget{lvv-t189---verify-implementation-of-base-facility-infrastructure-dms-req-0176}{\subsection{\texorpdfstring{\href{https://jira.lsstcorp.org/secure/Tests.jspa\#/testCase/LVV-T189}{LVV-T189}
- Verify implementation of Base Facility Infrastructure
(DMS-REQ-0176)}{LVV-T189 - Verify implementation of Base Facility Infrastructure (DMS-REQ-0176)}}\label{lvv-t189---verify-implementation-of-base-facility-infrastructure-dms-req-0176}}

\begin{longtable}[]{@{}llllll@{}}
\toprule
Version & Status & Priority & Verification Type & Critical Event &
Owner\tabularnewline
\midrule
\endhead
1 & Draft & Normal & Test & False & Robert Gruendl\tabularnewline
\bottomrule
\end{longtable}

\subsubsection{Test Items}\label{test-items-48}

Verify that the (a) planned infrastructure and (b) as-built
infrastructure for the Base Facility satisfies the needs for data
transfer and buffering, a copy of the Archive Facility, and support for
Commissioning.

\subsubsection{Requirements}\label{requirements-48}

\begin{itemize}
\tightlist
\item
  \href{https://jira.lsstcorp.org/browse/LVV-78}{LVV-78} -
  DMS-REQ-0176-V-01: Base Facility Infrastructure
\end{itemize}

\subsubsection{Test Script}\label{test-script-48}

\textbf{Step 1}\\
Analyze design and sizing model\\[2\baselineskip]

\hypertarget{lvv-t190---verify-implementation-of-base-facility-co-location-with-existing-facility-dms-req-0178}{\subsection{\texorpdfstring{\href{https://jira.lsstcorp.org/secure/Tests.jspa\#/testCase/LVV-T190}{LVV-T190}
- Verify implementation of Base Facility Co-Location with Existing
Facility
(DMS-REQ-0178)}{LVV-T190 - Verify implementation of Base Facility Co-Location with Existing Facility (DMS-REQ-0178)}}\label{lvv-t190---verify-implementation-of-base-facility-co-location-with-existing-facility-dms-req-0178}}

\begin{longtable}[]{@{}llllll@{}}
\toprule
Version & Status & Priority & Verification Type & Critical Event &
Owner\tabularnewline
\midrule
\endhead
1 & Draft & Normal & Test & False & Robert Gruendl\tabularnewline
\bottomrule
\end{longtable}

\subsubsection{Test Items}\label{test-items-49}

Verify that the Base Facility is located at an existing known supported
facility.

\subsubsection{Requirements}\label{requirements-49}

\begin{itemize}
\tightlist
\item
  \href{https://jira.lsstcorp.org/browse/LVV-80}{LVV-80} -
  DMS-REQ-0178-V-01: Base Facility Co-Location with Existing Facility
\end{itemize}

\subsubsection{Test Script}\label{test-script-49}

\textbf{Step 1}\\
Analyze design\\[2\baselineskip]

\hypertarget{lvv-t191---verify-implementation-of-commissioning-cluster-dms-req-0316}{\subsection{\texorpdfstring{\href{https://jira.lsstcorp.org/secure/Tests.jspa\#/testCase/LVV-T191}{LVV-T191}
- Verify implementation of Commissioning Cluster
(DMS-REQ-0316)}{LVV-T191 - Verify implementation of Commissioning Cluster (DMS-REQ-0316)}}\label{lvv-t191---verify-implementation-of-commissioning-cluster-dms-req-0316}}

\begin{longtable}[]{@{}llllll@{}}
\toprule
Version & Status & Priority & Verification Type & Critical Event &
Owner\tabularnewline
\midrule
\endhead
1 & Draft & Normal & Test & False & Robert Gruendl\tabularnewline
\bottomrule
\end{longtable}

\subsubsection{Test Items}\label{test-items-50}

Verify that the Commissioning Cluster has sufficient Compute/Storage/LAN
at the Base Facility to support Commissioning.

\subsubsection{Requirements}\label{requirements-50}

\begin{itemize}
\tightlist
\item
  \href{https://jira.lsstcorp.org/browse/LVV-147}{LVV-147} -
  DMS-REQ-0316-V-01: Commissioning Cluster
\end{itemize}

\subsubsection{Test Script}\label{test-script-50}

\textbf{Step 1}\\
Analyze design and budget\\[2\baselineskip]

\hypertarget{lvv-t192---verify-implementation-of-base-wireless-lan-wifi-dms-req-0352}{\subsection{\texorpdfstring{\href{https://jira.lsstcorp.org/secure/Tests.jspa\#/testCase/LVV-T192}{LVV-T192}
- Verify implementation of Base Wireless LAN (WiFi)
(DMS-REQ-0352)}{LVV-T192 - Verify implementation of Base Wireless LAN (WiFi) (DMS-REQ-0352)}}\label{lvv-t192---verify-implementation-of-base-wireless-lan-wifi-dms-req-0352}}

\begin{longtable}[]{@{}llllll@{}}
\toprule
Version & Status & Priority & Verification Type & Critical Event &
Owner\tabularnewline
\midrule
\endhead
1 & Draft & Normal & Test & False & Robert Gruendl\tabularnewline
\bottomrule
\end{longtable}

\subsubsection{Test Items}\label{test-items-51}

Verify (a) plannned and (b) as-built wireless network at the Base
Facility supports minBaseWiFi bandwidth.

\subsubsection{Requirements}\label{requirements-51}

\begin{itemize}
\tightlist
\item
  \href{https://jira.lsstcorp.org/browse/LVV-183}{LVV-183} -
  DMS-REQ-0352-V-01: Base Wireless LAN (WiFi)
\end{itemize}

\subsubsection{Test Script}\label{test-script-51}

\textbf{Step 1}\\
Delegate to Networks\\[2\baselineskip]

\hypertarget{lvv-t193---verify-implementation-of-base-to-archive-network-dms-req-0180}{\subsection{\texorpdfstring{\href{https://jira.lsstcorp.org/secure/Tests.jspa\#/testCase/LVV-T193}{LVV-T193}
- Verify implementation of Base to Archive Network
(DMS-REQ-0180)}{LVV-T193 - Verify implementation of Base to Archive Network (DMS-REQ-0180)}}\label{lvv-t193---verify-implementation-of-base-to-archive-network-dms-req-0180}}

\begin{longtable}[]{@{}llllll@{}}
\toprule
Version & Status & Priority & Verification Type & Critical Event &
Owner\tabularnewline
\midrule
\endhead
1 & Draft & Normal & Test & False & Robert Gruendl\tabularnewline
\bottomrule
\end{longtable}

\subsubsection{Test Items}\label{test-items-52}

Verify that the Base Facility can transfer a full image+metadata to the
Archive Center in baseToArchiveMaxTransferTime.

\subsubsection{Requirements}\label{requirements-52}

\begin{itemize}
\tightlist
\item
  \href{https://jira.lsstcorp.org/browse/LVV-81}{LVV-81} -
  DMS-REQ-0180-V-01: Base to Archive Network
\end{itemize}

\subsubsection{Test Script}\label{test-script-52}

\textbf{Step 1}\\
Delegate to Networks\\[2\baselineskip]

\hypertarget{lvv-t194---verify-implementation-of-base-to-archive-network-availability-dms-req-0181}{\subsection{\texorpdfstring{\href{https://jira.lsstcorp.org/secure/Tests.jspa\#/testCase/LVV-T194}{LVV-T194}
- Verify implementation of Base to Archive Network Availability
(DMS-REQ-0181)}{LVV-T194 - Verify implementation of Base to Archive Network Availability (DMS-REQ-0181)}}\label{lvv-t194---verify-implementation-of-base-to-archive-network-availability-dms-req-0181}}

\begin{longtable}[]{@{}llllll@{}}
\toprule
Version & Status & Priority & Verification Type & Critical Event &
Owner\tabularnewline
\midrule
\endhead
1 & Draft & Normal & Test & False & Robert Gruendl\tabularnewline
\bottomrule
\end{longtable}

\subsubsection{Test Items}\label{test-items-53}

Verify Network uptime between Base Facility and Archive Facility.

\subsubsection{Requirements}\label{requirements-53}

\begin{itemize}
\tightlist
\item
  \href{https://jira.lsstcorp.org/browse/LVV-82}{LVV-82} -
  DMS-REQ-0181-V-01: Base to Archive Network Availability
\end{itemize}

\subsubsection{Test Script}\label{test-script-53}

\textbf{Step 1}\\
Delegate to Networks\\[2\baselineskip]

\hypertarget{lvv-t195---verify-implementation-of-base-to-archive-network-reliability-dms-req-0182}{\subsection{\texorpdfstring{\href{https://jira.lsstcorp.org/secure/Tests.jspa\#/testCase/LVV-T195}{LVV-T195}
- Verify implementation of Base to Archive Network Reliability
(DMS-REQ-0182)}{LVV-T195 - Verify implementation of Base to Archive Network Reliability (DMS-REQ-0182)}}\label{lvv-t195---verify-implementation-of-base-to-archive-network-reliability-dms-req-0182}}

\begin{longtable}[]{@{}llllll@{}}
\toprule
Version & Status & Priority & Verification Type & Critical Event &
Owner\tabularnewline
\midrule
\endhead
1 & Draft & Normal & Test & False & Robert Gruendl\tabularnewline
\bottomrule
\end{longtable}

\subsubsection{Test Items}\label{test-items-54}

Verify uptime of the Base Facility to Archive Facility network.

\subsubsection{Requirements}\label{requirements-54}

\begin{itemize}
\tightlist
\item
  \href{https://jira.lsstcorp.org/browse/LVV-83}{LVV-83} -
  DMS-REQ-0182-V-01: Base to Archive Network Reliability
\end{itemize}

\subsubsection{Test Script}\label{test-script-54}

\textbf{Step 1}\\
Delegate to Networks\\[2\baselineskip]

\hypertarget{lvv-t196---verify-implementation-of-base-to-archive-network-secondary-link-dms-req-0183}{\subsection{\texorpdfstring{\href{https://jira.lsstcorp.org/secure/Tests.jspa\#/testCase/LVV-T196}{LVV-T196}
- Verify implementation of Base to Archive Network Secondary Link
(DMS-REQ-0183)}{LVV-T196 - Verify implementation of Base to Archive Network Secondary Link (DMS-REQ-0183)}}\label{lvv-t196---verify-implementation-of-base-to-archive-network-secondary-link-dms-req-0183}}

\begin{longtable}[]{@{}llllll@{}}
\toprule
Version & Status & Priority & Verification Type & Critical Event &
Owner\tabularnewline
\midrule
\endhead
1 & Draft & Normal & Test & False & Robert Gruendl\tabularnewline
\bottomrule
\end{longtable}

\subsubsection{Test Items}\label{test-items-55}

Verify the performance of a secondary network link meets needs for
operations support and catching up after outages.

\subsubsection{Requirements}\label{requirements-55}

\begin{itemize}
\tightlist
\item
  \href{https://jira.lsstcorp.org/browse/LVV-84}{LVV-84} -
  DMS-REQ-0183-V-01: Base to Archive Network Secondary Link
\end{itemize}

\subsubsection{Test Script}\label{test-script-55}

\textbf{Step 1}\\
Delegate to Networks\\[2\baselineskip]

\hypertarget{lvv-t197---verify-implementation-of-archive-center-dms-req-0185}{\subsection{\texorpdfstring{\href{https://jira.lsstcorp.org/secure/Tests.jspa\#/testCase/LVV-T197}{LVV-T197}
- Verify implementation of Archive Center
(DMS-REQ-0185)}{LVV-T197 - Verify implementation of Archive Center (DMS-REQ-0185)}}\label{lvv-t197---verify-implementation-of-archive-center-dms-req-0185}}

\begin{longtable}[]{@{}llllll@{}}
\toprule
Version & Status & Priority & Verification Type & Critical Event &
Owner\tabularnewline
\midrule
\endhead
1 & Draft & Normal & Test & False & Robert Gruendl\tabularnewline
\bottomrule
\end{longtable}

\subsubsection{Test Items}\label{test-items-56}

Verify that the Archive Center is sufficiently provisioned to support
prompt processing, DRP, and data access needs.

\subsubsection{Requirements}\label{requirements-56}

\begin{itemize}
\tightlist
\item
  \href{https://jira.lsstcorp.org/browse/LVV-85}{LVV-85} -
  DMS-REQ-0185-V-01: Archive Center
\end{itemize}

\subsubsection{Test Script}\label{test-script-56}

\textbf{Step 1}\\
Analyze design and sizing model\\[2\baselineskip]

\subsection{\texorpdfstring{\href{https://jira.lsstcorp.org/secure/Tests.jspa\#/testCase/LVV-T198}{LVV-T198}
- Verify implementation of Archive Center Disaster Recovery
(DMS-REQ-0186)}{LVV-T198 - Verify implementation of Archive Center Disaster Recovery (DMS-REQ-0186)}}\label{lvv-t198---verify-implementation-of-archive-center-disaster-recovery-dms-req-0186}

\begin{longtable}[]{@{}llllll@{}}
\toprule
Version & Status & Priority & Verification Type & Critical Event &
Owner\tabularnewline
\midrule
\endhead
1 & Draft & Normal & Test & False & Robert Gruendl\tabularnewline
\bottomrule
\end{longtable}

\subsubsection{Test Items}\label{test-items-57}

Verify disaster recovery plan for Archive Center.

\subsubsection{Requirements}\label{requirements-57}

\begin{itemize}
\tightlist
\item
  \href{https://jira.lsstcorp.org/browse/LVV-86}{LVV-86} -
  DMS-REQ-0186-V-01: Archive Center Disaster Recovery
\end{itemize}

\subsubsection{Test Script}\label{test-script-57}

\textbf{Step 1}\\
Analyze design; simulate storage failure, observe restore from disaster
recovery\\[2\baselineskip]

\hypertarget{lvv-t199---verify-implementation-of-archive-center-co-location-with-existing-facility-dms-req-0187}{\subsection{\texorpdfstring{\href{https://jira.lsstcorp.org/secure/Tests.jspa\#/testCase/LVV-T199}{LVV-T199}
- Verify implementation of Archive Center Co-Location with Existing
Facility
(DMS-REQ-0187)}{LVV-T199 - Verify implementation of Archive Center Co-Location with Existing Facility (DMS-REQ-0187)}}\label{lvv-t199---verify-implementation-of-archive-center-co-location-with-existing-facility-dms-req-0187}}

\begin{longtable}[]{@{}llllll@{}}
\toprule
Version & Status & Priority & Verification Type & Critical Event &
Owner\tabularnewline
\midrule
\endhead
1 & Draft & Normal & Test & False & Robert Gruendl\tabularnewline
\bottomrule
\end{longtable}

\subsubsection{Test Items}\label{test-items-58}

Verify the Archive Center is located at an existing supported facility.

\subsubsection{Requirements}\label{requirements-58}

\begin{itemize}
\tightlist
\item
  \href{https://jira.lsstcorp.org/browse/LVV-87}{LVV-87} -
  DMS-REQ-0187-V-01: Archive Center Co-Location with Existing Facility
\end{itemize}

\subsubsection{Test Script}\label{test-script-58}

\textbf{Step 1}\\
Analyze design\\[2\baselineskip]

\hypertarget{lvv-t200---verify-implementation-of-archive-to-data-access-center-network-dms-req-0188}{\subsection{\texorpdfstring{\href{https://jira.lsstcorp.org/secure/Tests.jspa\#/testCase/LVV-T200}{LVV-T200}
- Verify implementation of Archive to Data Access Center Network
(DMS-REQ-0188)}{LVV-T200 - Verify implementation of Archive to Data Access Center Network (DMS-REQ-0188)}}\label{lvv-t200---verify-implementation-of-archive-to-data-access-center-network-dms-req-0188}}

\begin{longtable}[]{@{}llllll@{}}
\toprule
Version & Status & Priority & Verification Type & Critical Event &
Owner\tabularnewline
\midrule
\endhead
1 & Draft & Normal & Test & False & Robert Gruendl\tabularnewline
\bottomrule
\end{longtable}

\subsubsection{Test Items}\label{test-items-59}

Verify sufficient bandwidth between Archive Center and Data Access
Centers of at least archToDacBandwidth.

\subsubsection{Requirements}\label{requirements-59}

\begin{itemize}
\tightlist
\item
  \href{https://jira.lsstcorp.org/browse/LVV-88}{LVV-88} -
  DMS-REQ-0188-V-01: Archive to Data Access Center Network
\end{itemize}

\subsubsection{Test Script}\label{test-script-59}

\textbf{Step 1}\\
Delegate to Networks\\[2\baselineskip]

\hypertarget{lvv-t201---verify-implementation-of-archive-to-data-access-center-network-availability-dms-req-0189}{\subsection{\texorpdfstring{\href{https://jira.lsstcorp.org/secure/Tests.jspa\#/testCase/LVV-T201}{LVV-T201}
- Verify implementation of Archive to Data Access Center Network
Availability
(DMS-REQ-0189)}{LVV-T201 - Verify implementation of Archive to Data Access Center Network Availability (DMS-REQ-0189)}}\label{lvv-t201---verify-implementation-of-archive-to-data-access-center-network-availability-dms-req-0189}}

\begin{longtable}[]{@{}llllll@{}}
\toprule
Version & Status & Priority & Verification Type & Critical Event &
Owner\tabularnewline
\midrule
\endhead
1 & Draft & Normal & Test & False & Robert Gruendl\tabularnewline
\bottomrule
\end{longtable}

\subsubsection{Requirements}\label{requirements-60}

\begin{itemize}
\tightlist
\item
  \href{https://jira.lsstcorp.org/browse/LVV-89}{LVV-89} -
  DMS-REQ-0189-V-01: Archive to Data Access Center Network Availability
\end{itemize}

\subsubsection{Test Script}\label{test-script-60}

\textbf{Step 1}\\
Delegate to Networks\\[2\baselineskip]

\hypertarget{lvv-t202---verify-implementation-of-archive-to-data-access-center-network-reliability-dms-req-0190}{\subsection{\texorpdfstring{\href{https://jira.lsstcorp.org/secure/Tests.jspa\#/testCase/LVV-T202}{LVV-T202}
- Verify implementation of Archive to Data Access Center Network
Reliability
(DMS-REQ-0190)}{LVV-T202 - Verify implementation of Archive to Data Access Center Network Reliability (DMS-REQ-0190)}}\label{lvv-t202---verify-implementation-of-archive-to-data-access-center-network-reliability-dms-req-0190}}

\begin{longtable}[]{@{}llllll@{}}
\toprule
Version & Status & Priority & Verification Type & Critical Event &
Owner\tabularnewline
\midrule
\endhead
1 & Draft & Normal & Test & False & Robert Gruendl\tabularnewline
\bottomrule
\end{longtable}

\subsubsection{Test Items}\label{test-items-60}

Verify the reliability of the Archive to Data Access Center
communications.

\subsubsection{Requirements}\label{requirements-61}

\begin{itemize}
\tightlist
\item
  \href{https://jira.lsstcorp.org/browse/LVV-90}{LVV-90} -
  DMS-REQ-0190-V-01: Archive to Data Access Center Network Reliability
\end{itemize}

\subsubsection{Test Script}\label{test-script-61}

\textbf{Step 1}\\
Delegate to Networks\\[2\baselineskip]

\hypertarget{lvv-t204---verify-implementation-of-access-to-catalogs-for-external-level-3-processing-dms-req-0122}{\subsection{\texorpdfstring{\href{https://jira.lsstcorp.org/secure/Tests.jspa\#/testCase/LVV-T204}{LVV-T204}
- Verify implementation of Access to catalogs for external Level 3
processing
(DMS-REQ-0122)}{LVV-T204 - Verify implementation of Access to catalogs for external Level 3 processing (DMS-REQ-0122)}}\label{lvv-t204---verify-implementation-of-access-to-catalogs-for-external-level-3-processing-dms-req-0122}}

\begin{longtable}[]{@{}llllll@{}}
\toprule
Version & Status & Priority & Verification Type & Critical Event &
Owner\tabularnewline
\midrule
\endhead
1 & Draft & Normal & Test & False & Kian-Tat Lim\tabularnewline
\bottomrule
\end{longtable}

\subsubsection{Test Items}\label{test-items-61}

Verify that catalog export, and maintenance/validation tools for Level 3
products to outside of the Data Access Centers.

\subsubsection{Requirements}\label{requirements-62}

\begin{itemize}
\tightlist
\item
  \href{https://jira.lsstcorp.org/browse/LVV-50}{LVV-50} -
  DMS-REQ-0122-V-01: Access to catalogs for external Level 3 processing
\end{itemize}

\subsubsection{Test Script}\label{test-script-62}

\textbf{Step 1}\\
Execute bulk distribution of DRP catalogs\\[2\baselineskip]\textbf{Step
2}\\
Observe correct transfer and use of maintenance/validation
tools\\[2\baselineskip]

\hypertarget{lvv-t205---verify-implementation-of-access-to-input-catalogs-for-dac-based-level-3-processing-dms-req-0123}{\subsection{\texorpdfstring{\href{https://jira.lsstcorp.org/secure/Tests.jspa\#/testCase/LVV-T205}{LVV-T205}
- Verify implementation of Access to input catalogs for DAC-based Level
3 processing
(DMS-REQ-0123)}{LVV-T205 - Verify implementation of Access to input catalogs for DAC-based Level 3 processing (DMS-REQ-0123)}}\label{lvv-t205---verify-implementation-of-access-to-input-catalogs-for-dac-based-level-3-processing-dms-req-0123}}

\begin{longtable}[]{@{}llllll@{}}
\toprule
Version & Status & Priority & Verification Type & Critical Event &
Owner\tabularnewline
\midrule
\endhead
1 & Draft & Normal & Test & False & Robert Gruendl\tabularnewline
\bottomrule
\end{longtable}

\subsubsection{Test Items}\label{test-items-62}

Verify that data products are available at the Data Access Centers for
use in Level 3 processing.

\subsubsection{Requirements}\label{requirements-63}

\begin{itemize}
\tightlist
\item
  \href{https://jira.lsstcorp.org/browse/LVV-51}{LVV-51} -
  DMS-REQ-0123-V-01: Access to input catalogs for DAC-based Level 3
  processing
\end{itemize}

\subsubsection{Test Script}\label{test-script-63}

\textbf{Step 1}\\
Load Prompt and DR catalogs into PDAC, observe access via
LSP\\[2\baselineskip]

\hypertarget{lvv-t207---verify-implementation-of-access-to-images-for-external-level-3-processing-dms-req-0126}{\subsection{\texorpdfstring{\href{https://jira.lsstcorp.org/secure/Tests.jspa\#/testCase/LVV-T207}{LVV-T207}
- Verify implementation of Access to images for external Level 3
processing
(DMS-REQ-0126)}{LVV-T207 - Verify implementation of Access to images for external Level 3 processing (DMS-REQ-0126)}}\label{lvv-t207---verify-implementation-of-access-to-images-for-external-level-3-processing-dms-req-0126}}

\begin{longtable}[]{@{}llllll@{}}
\toprule
Version & Status & Priority & Verification Type & Critical Event &
Owner\tabularnewline
\midrule
\endhead
1 & Draft & Normal & Test & False & Kian-Tat Lim\tabularnewline
\bottomrule
\end{longtable}

\subsubsection{Test Items}\label{test-items-63}

Verify that bulk distribution of images, and accompanying
maintenance/validation tools for Level 3 image products to outside of
the Data Access Centers.

\subsubsection{Requirements}\label{requirements-64}

\begin{itemize}
\tightlist
\item
  \href{https://jira.lsstcorp.org/browse/LVV-54}{LVV-54} -
  DMS-REQ-0126-V-01: Access to images for external Level 3 processing
\end{itemize}

\subsubsection{Test Script}\label{test-script-64}

\textbf{Step 1}\\
Execute bulk distribution of DRP images\\[2\baselineskip]\textbf{Step
2}\\
Observe correct transfer and use of maintenance/validation
tools\\[2\baselineskip]

\hypertarget{lvv-t208---verify-implementation-of-access-to-input-images-for-dac-based-level-3-processing-dms-req-0127}{\subsection{\texorpdfstring{\href{https://jira.lsstcorp.org/secure/Tests.jspa\#/testCase/LVV-T208}{LVV-T208}
- Verify implementation of Access to input images for DAC-based Level 3
processing
(DMS-REQ-0127)}{LVV-T208 - Verify implementation of Access to input images for DAC-based Level 3 processing (DMS-REQ-0127)}}\label{lvv-t208---verify-implementation-of-access-to-input-images-for-dac-based-level-3-processing-dms-req-0127}}

\begin{longtable}[]{@{}llllll@{}}
\toprule
Version & Status & Priority & Verification Type & Critical Event &
Owner\tabularnewline
\midrule
\endhead
1 & Draft & Normal & Test & False & Kian-Tat Lim\tabularnewline
\bottomrule
\end{longtable}

\subsubsection{Test Items}\label{test-items-64}

Verify that prompt processing and DRP products are available at the DACs
for Level 3 processing at the DACs.

\subsubsection{Requirements}\label{requirements-65}

\begin{itemize}
\tightlist
\item
  \href{https://jira.lsstcorp.org/browse/LVV-55}{LVV-55} -
  DMS-REQ-0127-V-01: Access to input images for DAC-based Level 3
  processing
\end{itemize}

\subsubsection{Test Script}\label{test-script-65}

\textbf{Step 1}\\
Load Prompt and DR images into PDAC\\[2\baselineskip]\textbf{Step 2}\\
Observe access via LSP\\[2\baselineskip]

\hypertarget{lvv-t212---verify-implementation-of-no-limit-on-data-access-centers-dms-req-0197}{\subsection{\texorpdfstring{\href{https://jira.lsstcorp.org/secure/Tests.jspa\#/testCase/LVV-T212}{LVV-T212}
- Verify implementation of No Limit on Data Access Centers
(DMS-REQ-0197)}{LVV-T212 - Verify implementation of No Limit on Data Access Centers (DMS-REQ-0197)}}\label{lvv-t212---verify-implementation-of-no-limit-on-data-access-centers-dms-req-0197}}

\begin{longtable}[]{@{}llllll@{}}
\toprule
Version & Status & Priority & Verification Type & Critical Event &
Owner\tabularnewline
\midrule
\endhead
1 & Draft & Normal & Test & False & Colin Slater\tabularnewline
\bottomrule
\end{longtable}

\subsubsection{Test Items}\label{test-items-65}

Verify that additional Data Access Centers can be set up.

\subsubsection{Requirements}\label{requirements-66}

\begin{itemize}
\tightlist
\item
  \href{https://jira.lsstcorp.org/browse/LVV-95}{LVV-95} -
  DMS-REQ-0197-V-01: No Limit on Data Access Centers
\end{itemize}

\subsubsection{Test Script}\label{test-script-66}

\textbf{Step 1}\\
Analyze design; instantiate and load simulated DAC, observe correct
functioning\\[2\baselineskip]

\section{Requirements Traceability}\label{requirements-traceability}

\begin{longtable}[]{p{13cm}p{3cm}}
\toprule
Requirements & Test Cases\tabularnewline
\midrule
\endhead
\href{https://jira.lsstcorp.org/browse/LVV-5}{LVV-5 - DMS-REQ-0008-V-01:
Pipeline Availability} &
\protect\hyperlink{lvv-t171---verify-implementation-of-pipeline-availability-dms-req-0008}{LVV-T171}\tabularnewline
\href{https://jira.lsstcorp.org/browse/LVV-6}{LVV-6 - DMS-REQ-0009-V-01:
Simulated Data} &
\protect\hyperlink{lvv-t125---verify-implementation-of-simulated-data-dms-req-0009}{LVV-T125}\tabularnewline
\href{https://jira.lsstcorp.org/browse/LVV-14}{LVV-14 -
DMS-REQ-0032-V-01: Image Differencing} &
\protect\hyperlink{lvv-t126---verify-implementation--image-differencing-dms-req-0032}{LVV-T126}\tabularnewline
\href{https://jira.lsstcorp.org/browse/LVV-15}{LVV-15 -
DMS-REQ-0033-V-01: Provide Source Detection Software} &
\protect\hyperlink{lvv-t127---verify-implementation-of-provide-source-detection-software-dms-req-0033}{LVV-T127}\tabularnewline
\href{https://jira.lsstcorp.org/browse/LVV-17}{LVV-17 -
DMS-REQ-0042-V-01: Provide Astrometric Model} &
\protect\hyperlink{lvv-t128---verify-implementation-provide-astrometric-model-dms-req-0042}{LVV-T128}\tabularnewline
\href{https://jira.lsstcorp.org/browse/LVV-18}{LVV-18 -
DMS-REQ-0043-V-01: Provide Calibrated Photometry} &
\protect\hyperlink{lvv-t129---verify-implementation-of-provide-calibrated-photometry-dms-req-0043}{LVV-T129}\tabularnewline
\href{https://jira.lsstcorp.org/browse/LVV-21}{LVV-21 -
DMS-REQ-0052-V-01: Enable a Range of Shape Measurement Approaches} &
\protect\hyperlink{lvv-t130---verify-implementation-of-enable-a-range-of-shape-measurement-approaches-dms-req-0052}{LVV-T130}\tabularnewline
\href{https://jira.lsstcorp.org/browse/LVV-27}{LVV-27 -
DMS-REQ-0065-V-01: Provide Image Access Services} &
\protect\hyperlink{lvv-t134---verify-implementation-of-provide-image-access-services-dms-req-0065}{LVV-T134}\tabularnewline
\href{https://jira.lsstcorp.org/browse/LVV-33}{LVV-33 -
DMS-REQ-0075-V-01: Catalog Queries} &
\protect\hyperlink{lvv-t149---verify-implementation-of-catalog-queries-dms-req-0075}{LVV-T149}\tabularnewline
\href{https://jira.lsstcorp.org/browse/LVV-34}{LVV-34 -
DMS-REQ-0077-V-01: Maintain Archive Publicly Accessible} &
\protect\hyperlink{lvv-t150---verify-implementation-of-maintain-archive-publicly-accessible-dms-req-0077}{LVV-T150}\tabularnewline
\href{https://jira.lsstcorp.org/browse/LVV-35}{LVV-35 -
DMS-REQ-0078-V-01: Catalog Export Formats} &
\protect\hyperlink{lvv-t151---verify-implementation-of-catalog-export-formats-dms-req-0078}{LVV-T151}\tabularnewline
\href{https://jira.lsstcorp.org/browse/LVV-37}{LVV-37 -
DMS-REQ-0094-V-01: Keep Historical Alert Archive} &
\protect\hyperlink{lvv-t152---verify-implementation-of-keep-historical-alert-archive-dms-req-0094}{LVV-T152}\tabularnewline
\href{https://jira.lsstcorp.org/browse/LVV-50}{LVV-50 -
DMS-REQ-0122-V-01: Access to catalogs for external Level 3 processing} &
\protect\hyperlink{lvv-t204---verify-implementation-of-access-to-catalogs-for-external-level-3-processing-dms-req-0122}{LVV-T204}\tabularnewline
\href{https://jira.lsstcorp.org/browse/LVV-51}{LVV-51 -
DMS-REQ-0123-V-01: Access to input catalogs for DAC-based Level 3
processing} &
\protect\hyperlink{lvv-t205---verify-implementation-of-access-to-input-catalogs-for-dac-based-level-3-processing-dms-req-0123}{LVV-T205}\tabularnewline
\href{https://jira.lsstcorp.org/browse/LVV-53}{LVV-53 -
DMS-REQ-0125-V-01: Software framework for Level 3 catalog processing} &
\protect\hyperlink{lvv-t120---verify-implementation-of-software-framework-for-level-3-catalog-processing-dms-req-0125}{LVV-T120}\tabularnewline
\href{https://jira.lsstcorp.org/browse/LVV-54}{LVV-54 -
DMS-REQ-0126-V-01: Access to images for external Level 3 processing} &
\protect\hyperlink{lvv-t207---verify-implementation-of-access-to-images-for-external-level-3-processing-dms-req-0126}{LVV-T207}\tabularnewline
\href{https://jira.lsstcorp.org/browse/LVV-55}{LVV-55 -
DMS-REQ-0127-V-01: Access to input images for DAC-based Level 3
processing} &
\protect\hyperlink{lvv-t208---verify-implementation-of-access-to-input-images-for-dac-based-level-3-processing-dms-req-0127}{LVV-T208}\tabularnewline
\href{https://jira.lsstcorp.org/browse/LVV-56}{LVV-56 -
DMS-REQ-0128-V-01: Software framework for Level 3 image processing} &
\protect\hyperlink{lvv-t121---verify-implementation-of-software-framework-for-level-3-image-processing-dms-req-0128}{LVV-T121}\tabularnewline
\href{https://jira.lsstcorp.org/browse/LVV-60}{LVV-60 -
DMS-REQ-0155-V-01: Provide Data Access Services} &
\protect\hyperlink{lvv-t135---verify-implementation-of-provide-data-access-services-dms-req-0155}{LVV-T135}\tabularnewline
\href{https://jira.lsstcorp.org/browse/LVV-62}{LVV-62 -
DMS-REQ-0158-V-01: Provide Pipeline Construction Services} &
\protect\hyperlink{lvv-t143---verify-implementation-of-provide-pipeline-construction-services-dms-req-0158}{LVV-T143}\tabularnewline
\href{https://jira.lsstcorp.org/browse/LVV-64}{LVV-64 -
DMS-REQ-0161-V-01: Optimization of Cost, Reliability and Availability in
Order} &
\protect\hyperlink{lvv-t172---verify-implementation-of-optimization-of-cost-reliability-and-availability-dms-req-0161}{LVV-T172}\tabularnewline
\href{https://jira.lsstcorp.org/browse/LVV-65}{LVV-65 -
DMS-REQ-0162-V-01: Pipeline Throughput} &
\protect\hyperlink{lvv-t173---verify-implementation-of-pipeline-throughput-dms-req-0162}{LVV-T173}\tabularnewline
\href{https://jira.lsstcorp.org/browse/LVV-66}{LVV-66 -
DMS-REQ-0163-V-01: Re-processing Capacity} &
\protect\hyperlink{lvv-t174---verify-implementation-of-re-processing-capacity-dms-req-0163}{LVV-T174}\tabularnewline
\href{https://jira.lsstcorp.org/browse/LVV-71}{LVV-71 -
DMS-REQ-0168-V-01: Summit Facility Data Communications} &
\protect\hyperlink{lvv-t181---verify-implementation-of-summit-facility-data-communications-dms-req-0168}{LVV-T181}\tabularnewline
\href{https://jira.lsstcorp.org/browse/LVV-72}{LVV-72 -
DMS-REQ-0170-V-01: Prefer Computing and Storage Down} &
\protect\hyperlink{lvv-t182---verify-implementation-of-prefer-computing-and-storage-down-dms-req-0170}{LVV-T182}\tabularnewline
\href{https://jira.lsstcorp.org/browse/LVV-73}{LVV-73 -
DMS-REQ-0171-V-01: Summit to Base Network} &
\protect\hyperlink{lvv-t184---verify-implementation-of-summit-to-base-network-dms-req-0171}{LVV-T184}\tabularnewline
\href{https://jira.lsstcorp.org/browse/LVV-74}{LVV-74 -
DMS-REQ-0172-V-01: Summit to Base Network Availability} &
\protect\hyperlink{lvv-t185---verify-implementation-of-summit-to-base-network-availability-dms-req-0172}{LVV-T185}\tabularnewline
\href{https://jira.lsstcorp.org/browse/LVV-75}{LVV-75 -
DMS-REQ-0173-V-01: Summit to Base Network Reliability} &
\protect\hyperlink{lvv-t186---verify-implementation-of-summit-to-base-network-reliability-dms-req-0173}{LVV-T186}\tabularnewline
\href{https://jira.lsstcorp.org/browse/LVV-76}{LVV-76 -
DMS-REQ-0174-V-01: Summit to Base Network Secondary Link} &
\protect\hyperlink{lvv-t187---verify-implementation-of-summit-to-base-network-secondary-link-dms-req-0174}{LVV-T187}\tabularnewline
\href{https://jira.lsstcorp.org/browse/LVV-77}{LVV-77 -
DMS-REQ-0175-V-01: Summit to Base Network Ownership and Operation} &
\protect\hyperlink{lvv-t188---verify-implementation-of-summit-to-base-network-ownership-and-operation-dms-req-0175}{LVV-T188}\tabularnewline
\href{https://jira.lsstcorp.org/browse/LVV-78}{LVV-78 -
DMS-REQ-0176-V-01: Base Facility Infrastructure} &
\protect\hyperlink{lvv-t189---verify-implementation-of-base-facility-infrastructure-dms-req-0176}{LVV-T189}\tabularnewline
\href{https://jira.lsstcorp.org/browse/LVV-80}{LVV-80 -
DMS-REQ-0178-V-01: Base Facility Co-Location with Existing Facility} &
\protect\hyperlink{lvv-t190---verify-implementation-of-base-facility-co-location-with-existing-facility-dms-req-0178}{LVV-T190}\tabularnewline
\href{https://jira.lsstcorp.org/browse/LVV-81}{LVV-81 -
DMS-REQ-0180-V-01: Base to Archive Network} &
\protect\hyperlink{lvv-t193---verify-implementation-of-base-to-archive-network-dms-req-0180}{LVV-T193}\tabularnewline
\href{https://jira.lsstcorp.org/browse/LVV-82}{LVV-82 -
DMS-REQ-0181-V-01: Base to Archive Network Availability} &
\protect\hyperlink{lvv-t194---verify-implementation-of-base-to-archive-network-availability-dms-req-0181}{LVV-T194}\tabularnewline
\href{https://jira.lsstcorp.org/browse/LVV-83}{LVV-83 -
DMS-REQ-0182-V-01: Base to Archive Network Reliability} &
\protect\hyperlink{lvv-t195---verify-implementation-of-base-to-archive-network-reliability-dms-req-0182}{LVV-T195}\tabularnewline
\href{https://jira.lsstcorp.org/browse/LVV-84}{LVV-84 -
DMS-REQ-0183-V-01: Base to Archive Network Secondary Link} &
\protect\hyperlink{lvv-t196---verify-implementation-of-base-to-archive-network-secondary-link-dms-req-0183}{LVV-T196}\tabularnewline
\href{https://jira.lsstcorp.org/browse/LVV-85}{LVV-85 -
DMS-REQ-0185-V-01: Archive Center} &
\protect\hyperlink{lvv-t197---verify-implementation-of-archive-center-dms-req-0185}{LVV-T197}\tabularnewline
\href{https://jira.lsstcorp.org/browse/LVV-86}{LVV-86 -
DMS-REQ-0186-V-01: Archive Center Disaster Recovery} &
\protect\hyperlink{lvv-t198---verify-implementation-of--archive-center-disaster-recovery-dms-req-0186}{LVV-T198}\tabularnewline
\href{https://jira.lsstcorp.org/browse/LVV-87}{LVV-87 -
DMS-REQ-0187-V-01: Archive Center Co-Location with Existing Facility} &
\protect\hyperlink{lvv-t199---verify-implementation-of-archive-center-co-location-with-existing-facility-dms-req-0187}{LVV-T199}\tabularnewline
\href{https://jira.lsstcorp.org/browse/LVV-88}{LVV-88 -
DMS-REQ-0188-V-01: Archive to Data Access Center Network} &
\protect\hyperlink{lvv-t200---verify-implementation-of-archive-to-data-access-center-network-dms-req-0188}{LVV-T200}\tabularnewline
\href{https://jira.lsstcorp.org/browse/LVV-89}{LVV-89 -
DMS-REQ-0189-V-01: Archive to Data Access Center Network Availability} &
\protect\hyperlink{lvv-t201---verify-implementation-of-archive-to-data-access-center-network-availability-dms-req-0189}{LVV-T201}\tabularnewline
\href{https://jira.lsstcorp.org/browse/LVV-90}{LVV-90 -
DMS-REQ-0190-V-01: Archive to Data Access Center Network Reliability} &
\protect\hyperlink{lvv-t202---verify-implementation-of-archive-to-data-access-center-network-reliability-dms-req-0190}{LVV-T202}\tabularnewline
\href{https://jira.lsstcorp.org/browse/LVV-95}{LVV-95 -
DMS-REQ-0197-V-01: No Limit on Data Access Centers} &
\protect\hyperlink{lvv-t212---verify-implementation-of-no-limit-on-data-access-centers-dms-req-0197}{LVV-T212}\tabularnewline
\href{https://jira.lsstcorp.org/browse/LVV-127}{LVV-127 -
DMS-REQ-0296-V-01: Pre-cursor, and Real Data} &
\protect\hyperlink{lvv-t132---verify-implementation-of-pre-cursor-and-real-data-dms-req-0296}{LVV-T132}\tabularnewline
\href{https://jira.lsstcorp.org/browse/LVV-129}{LVV-129 -
DMS-REQ-0298-V-01: Data Product and Raw Data Access} &
\protect\hyperlink{lvv-t136---verify-implementation-of-data-product-and-raw-data-access-dms-req-0298}{LVV-T136}\tabularnewline
\href{https://jira.lsstcorp.org/browse/LVV-130}{LVV-130 -
DMS-REQ-0299-V-01: Data Product Ingest} &
\protect\hyperlink{lvv-t137---verify-implementation-of-data-product-ingest-dms-req-0299}{LVV-T137}\tabularnewline
\href{https://jira.lsstcorp.org/browse/LVV-136}{LVV-136 -
DMS-REQ-0305-V-01: Task Specification} &
\protect\hyperlink{lvv-t144---verify-implementation-of-task-specification-dms-req-0305}{LVV-T144}\tabularnewline
\href{https://jira.lsstcorp.org/browse/LVV-138}{LVV-138 -
DMS-REQ-0307-V-01: Unique Processing Coverage} &
\protect\hyperlink{lvv-t148---verify-implementation-of-unique-processing-coverage-dms-req-0307}{LVV-T148}\tabularnewline
\href{https://jira.lsstcorp.org/browse/LVV-140}{LVV-140 -
DMS-REQ-0309-V-01: Raw Data Archiving Reliability} &
\protect\hyperlink{lvv-t154---verify-implementation-of-raw-data-archiving-reliability-dms-req-0309}{LVV-T154}\tabularnewline
\href{https://jira.lsstcorp.org/browse/LVV-141}{LVV-141 -
DMS-REQ-0310-V-01: Un-Archived Data Product Cache} &
\protect\hyperlink{lvv-t155---verify-implementation-of-un-archived-data-product-cache-dms-req-0310}{LVV-T155}\tabularnewline
\href{https://jira.lsstcorp.org/browse/LVV-143}{LVV-143 -
DMS-REQ-0312-V-01: Level 1 Data Product Access} &
\protect\hyperlink{lvv-t157---verify-implementation-level-1-data-product-access-dms-req-0312}{LVV-T157}\tabularnewline
\href{https://jira.lsstcorp.org/browse/LVV-144}{LVV-144 -
DMS-REQ-0313-V-01: Level 1 \& 2 Catalog Access} &
\protect\hyperlink{lvv-t158---verify-implementation-level-1-and-2-catalog-access-dms-req-0313}{LVV-T158}\tabularnewline
\href{https://jira.lsstcorp.org/browse/LVV-146}{LVV-146 -
DMS-REQ-0315-V-01: DMS Communication with OCS} &
\protect\hyperlink{lvv-t183---verify-implementation-of-dms-communication-with-ocs-dms-req-0315}{LVV-T183}\tabularnewline
\href{https://jira.lsstcorp.org/browse/LVV-147}{LVV-147 -
DMS-REQ-0316-V-01: Commissioning Cluster} &
\protect\hyperlink{lvv-t191---verify-implementation-of-commissioning-cluster-dms-req-0316}{LVV-T191}\tabularnewline
\href{https://jira.lsstcorp.org/browse/LVV-149}{LVV-149 -
DMS-REQ-0318-V-01: Data Management Unscheduled Downtime} &
\protect\hyperlink{lvv-t180---verify-implementation-of-data-management-unscheduled-downtime-dms-req-0318}{LVV-T180}\tabularnewline
\href{https://jira.lsstcorp.org/browse/LVV-167}{LVV-167 -
DMS-REQ-0336-V-01: Regenerating Data Products from Previous Data
Releases} &
\protect\hyperlink{lvv-t159---verify-implementation-of-regenerating-data-products-from-previous-data-releases-dms-req-0336}{LVV-T159}\tabularnewline
\href{https://jira.lsstcorp.org/browse/LVV-171}{LVV-171 -
DMS-REQ-0340-V-01: Access Controls of Level 3 Data Products} &
\protect\hyperlink{lvv-t123---verify-implementation-of-access-controls-of-level-3-data-products-dms-req-0340}{LVV-T123}\tabularnewline
\href{https://jira.lsstcorp.org/browse/LVV-172}{LVV-172 -
DMS-REQ-0341-V-01: Providing a Precovery Service} &
\protect\hyperlink{lvv-t160---verify-implementation-of-providing-a-precovery-service-dms-req-0341}{LVV-T160}\tabularnewline
\href{https://jira.lsstcorp.org/browse/LVV-176}{LVV-176 -
DMS-REQ-0345-V-01: Logging of catalog queries} &
\protect\hyperlink{lvv-t161---verify-implementation-of-logging-of-catalog-queries-dms-req-0345}{LVV-T161}\tabularnewline
\href{https://jira.lsstcorp.org/browse/LVV-182}{LVV-182 -
DMS-REQ-0351-V-01: Provide Beam Projector Coordinate Calculation
Software} &
\protect\hyperlink{lvv-t133---verify-implementation-of-provide-beam-projector-coordinate-calculation-software-dms-req-0351}{LVV-T133}\tabularnewline
\href{https://jira.lsstcorp.org/browse/LVV-183}{LVV-183 -
DMS-REQ-0352-V-01: Base Wireless LAN (WiFi)} &
\protect\hyperlink{lvv-t192---verify-implementation-of-base-wireless-lan-wifi-dms-req-0352}{LVV-T192}\tabularnewline
\href{https://jira.lsstcorp.org/browse/LVV-190}{LVV-190 -
DMS-REQ-0364-V-01: Data Access Services} &
\protect\hyperlink{lvv-t163---verify-implementation-of-data-access-services-dms-req-0364}{LVV-T163}\tabularnewline
\href{https://jira.lsstcorp.org/browse/LVV-191}{LVV-191 -
DMS-REQ-0365-V-01: Operations Subsets} &
\protect\hyperlink{lvv-t164---verify-implementation-of-operations-subsets-dms-req-0365}{LVV-T164}\tabularnewline
\href{https://jira.lsstcorp.org/browse/LVV-192}{LVV-192 -
DMS-REQ-0366-V-01: Subsets Support} &
\protect\hyperlink{lvv-t165---verify-implementation-of-subsets-support-dms-req-0366}{LVV-T165}\tabularnewline
\href{https://jira.lsstcorp.org/browse/LVV-195}{LVV-195 -
DMS-REQ-0369-V-01: Evolution} &
\protect\hyperlink{lvv-t168---verify-design-of-data-access-services-allows-evolution-of-the-lsst-data-model-dms-req-0369}{LVV-T168}\tabularnewline
\href{https://jira.lsstcorp.org/browse/LVV-196}{LVV-196 -
DMS-REQ-0370-V-01: Older Release Behavior} &
\protect\hyperlink{lvv-t169---verify-implementation-of-older-release-behavior-dms-req-0370}{LVV-T169}\tabularnewline
\href{https://jira.lsstcorp.org/browse/LVV-197}{LVV-197 -
DMS-REQ-0371-V-01: Query Availability} &
\protect\hyperlink{lvv-t170---verify-implementation-of-query-availability-dms-req-0371}{LVV-T170}\tabularnewline
\bottomrule
\end{longtable}




\end{document}
